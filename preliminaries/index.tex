В этом разделе вводятся основные понятия теории --- аффинные суперсхемы,
замкнутые подсуперсхемы

\begin{subsection}{$\K$-функторы}
Определения, данные в ~\cite{jantzen} для обычного случая, можно почти дословно
перенести на ~суперслучай. Некоторые из ~них можно найти в ~\cite{affine_quotients}.

Введем некоторые предварительные обозначения. $ \K $ -- произвольное поле,
$ \SAlg $ --- категория супералгебр над ~полем $ \K $,
$ \Sets $ --- категория множеств, $ \Groups $ --- категория групп.

\begin{definition}
  $\K$-функтором назовем функтор из~категории $ \SAlg $ в~$ \Sets $.
\end{definition}

Для $\K$-функторов $ X, X' $ обозначим через $ \Mor(X, X') $ множество морфизмов из $ Х $ в $ X' $.

\begin{definition}
  Пусть $ X $ --- $\K$-функтор. $\K$-функтор $ Y $ называется
  подфунктором функтора $ X $, если $ ~\forall ~A, A' \in \SAlg
  \;~\forall ~\p \in \HomSAlg(A, A') $ выполнены условия:
  $ Y(A) \subset X(A) $ и $ Y(\p) = X(\p)|_{Y(A)} $.
\end{definition}

Для любого семейства подфункторов $ \{Y_i\}_{i \in I} \subset X $ определим
функтор-пересечение $ \bigcap_{\substack{i \in I}} Y_i $ следующим образом:
$$ ( \bigcap_{\substack{i \in I}} Y_i )(A) = \bigcap_{\substack{i \in I}} Y_i (A). $$

Для ~$ f \in \Mor(X, X') ~\forall ~Y' \subseteq X' $ определим функтор-прообраз
$$ (f^{-1}(Y'))(A) = f(A)^{-1} (Y'(A)) \qquad \text{для} ~A \in \SAlg. $$

Очевидно, что $ \bigcap_{i \in I}Y_i $ и $ f^{-1}(Y') $.
%
\begin{definition}
  Прямым произведением $\K$-функторов $ X_1$ и $ X_2 $ называется функтор
  $ (X_1 \times X_2)(A) = X_1(A) \times X_2(A) ~\;\text{для} ~A \in \SAlg $.
\end{definition}
 
\end{subsection}

\begin{subsection}{Аффинные суперсхемы}
  \begin{definition}
    $\K$-функтор $ \SSp R $, определенный как
    $$ (\SSp R)(A) = \HomSAlg(R, A) \qquad \text{для} ~A \in \SAlg, $$
    называется аффинной суперсхемой. Супералгебра $ R \in \SAlg $ называется координатной
    супералгеброй суперсхемы $ \SSp R $. Если $ X = \SSp R $, то $ R $ обозначается $ \K[X] $.
  \end{definition}

  \begin{lemma}[лемма Йонеды] \label{yoneda}
    $ \forall ~R \in \SAlg ~\forall ~\K$-функторa $ X $ отображение
    $ f \mapsto f(R)(id_R) $ является биекцией $$ \Mor(\SSp R, X) \simeq X(R). $$
    \proof {
      \qedhere
    }
  \end{lemma}

  \begin{definition}
    Аффинная суперсхема $ \A^{m|n} = \SSp \K[t_1, \ldots, t_m | z_1, \ldots, z_n] $
    называется $(m|n)$-аффинным суперпространством.
  \end{definition}
  Очевидно, что $ \A^{m|n} (B) = B_0^m \oplus B_1^n ~\text{для} ~B \in \SAlg $.
  В частности, $ \A^{1|1}(B) = B $ для любой супералгебры $ B $.

  \begin{definition}
    Пусть $ I $ --- суперидеал $ B \in \SAlg $. Подфунктор
    $ V(I) = \{ \p \in (\SSp R)(A) ~| ~\p(I) = 0 \} $ функтора $ \SSp R $
    называется замкнутым подфунктором, соответствующим суперидеалу $ I $.
  \end{definition}
  Очевидно, что $ V(I) \simeq \SSp (\K[X] / I) $.

  \begin{definition}
    Аффинная суперсхема называется алгебраической, если
    $ \K[X] \simeq \K[t_1, \ldots, t_m|z_1, \ldots, z_n] / I $ для некоторых
    $ m, n \in \N $ и конечнопорожденного суперидеала $ I $.
  \end{definition}
  \begin{definition}
    Аффинная суперсхема $ X $ называется редуцированной, если
    $ \K[X] $ не содержит нильпотентных элементов, отличных от 0.
  \end{definition}

\end{subsection}

\begin{subsection}{Групповые $\K$-функторы и аффинные групповые суперсхемы}
  \begin{definition}
    Групповым $\K$-функтором будем называть функтор из $ \SAlg $ в $ \Groups $.
  \end{definition}
  Если взять композицию группового функтора с забывающим функтором из $ \Groups $ в $ \Sets $,
  то групповой $\K$-функтор можно рассматривать как $\K$-функтор. Поэтому все результаты
  для $\K$-функторов можно перенести на групповые $\K$-функторы.

  Пусть $ G, H $ --- групповые $\K$-функторы. Обозначим через
  $ \Mor(G, H) $ множество морфизмов из $ G $ в $ H $, если рассматривать $ G $ и $ H $
  как $\K$-функторы;
  через $ \Hom(G, H) $ множество морфизмов групповых функторов.

  
\end{subsection}



