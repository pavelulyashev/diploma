Связная компонента, связная супергруппа, 
утверждение про центр группы (если оно нужно для доказательства).
\cite{affine_quotients}

Везде в этом пункте $ G $ -- аффинная групповая суперсхема над полем $ \K $.

\definition {
  Подфунктор $ \Z(G) $ групового $ \K $-функтора $ G $ называется центральным,
  если $ H $ -- подфунктор в $ G $ и $ \forall A \in \SAlg \; H(A) $ --
  центральная подгруппа в $ G(A) $.
}

\proposition \label{Z(G) closed in G} {
  Пусть $ G $ -- аффинная групповая суперсхема.
  $ \Z(G) $ -- замкнутая аффинная групповая подсуперсхема в $ G $.
}
\proof {
  
  \qedhere
}

\proposition \label{Lie(Z(G)) = Z(Lie(G))} {
  Если $ G $ связна, $ \charc K = 0 $, то $ \Lie(\Z(G)) = \Z(\Lie(G)) $.
}
\proof {
  
  \qedhere
}



\theorem \label{Exists H in G: Lie(H) = I} {
  Пусть $ \charc K = 0 $, $ G $ -- связная аффинная групповая суперсхема, \\
  $ I $ -- максимальный абелев суперидеал в $ \Lie(G) $.
  Существует $ H \lhd G : \Lie(H) = I $.
}
\proof {
  Обозначим $ L = \Lie(G) $. 
  Доказательство проведем индукцией по $ \dim L $. Предположим,
  что если $ H $  -- связная аффинная групповая суперсхема и $ \dim \Lie(H) < \dim L $,
  то утверждение выполнено для $ H $.

  Рассмотрим действие $ \Ad: G \to \GL(I) $, $ \ker \Ad = R $.
  Пусть $ J = \Lie(R) = \{ x \in L | [x, I] = 0 \} $.
  Очевидно, $ I \subseteq J $.

  Если $ \dim J \leqslant \dim L $, то по предположению индукции утверждение выполнено для $ R $,
  т.е. $ \exists \; H \lhd R : \Lie(H) = I $. Поскольку $ H \lhd R $
  и $ R \lhd G $ как ядро $ \Ad $, то $ H \lhd G $, следовательно, утверждение выполнено для $ G $.

  Рассмотрим случай $ \dim J = \dim L $.
  % J = L
  Т.к. $ G $ алгебраическая, то $ \dim L < \infty \hence J = L $.
  % I ⊆ Z(L)
  Отсюда следует, что $ [L, I] = 0 $, а в силу определения центра $ I \subseteq \Z(L) $.
  % I = Z(L)
  По условию $ I $ -- максимальный суперидеал $ \hence $ $ I $ не может быть
  собственным подмножеством $ \hence $ $ I = \Z(L) $.
  % I = Lie(Z(G))
  По лемме \ref{Lie(Z(G)) = Z(Lie(G))} получаем, что $ I = \Lie(\Z(G)) $, а $ \Z(G) \lhd G $.
\qedhere
}