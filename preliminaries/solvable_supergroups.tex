Для того, чтобы сформулировать определение разрешимой супергруппы,
сначала необходимо определить коммутант супергруппы.

Пусть $ S $ - алгебраическая матричная супергруппа. Рассмотрим отображение
$ S ~\times ~S ~\to ~S $, переводящее $ (x, y) $ в $ xyx^{-1}y^{-1} $.
Ядро $ I_{1} $ соотвествующего отображения $ K[S] ~\to ~K[S] ~\otimes ~K[S] $ состоит из функций, зануляющихся на всех коммутаторах из $ S $;
таким образом, замкнутое множество, им определяемое, является замыканием коммутаторов. Аналогично имеем отображение $ S^{2n} \to S $, 
переводящее $ (x_1, y_1, \ldots, x_n, y_n) $ в 
$ x_1y_1x_1^{-1}y_1^{-1} \cdots x_ny_nx_n^{-1}y_n^{-1} $.
Соответствующее отображение $ K[S] ~\to ~\otimes^{2n} ~K[S] $ имеет ядро
$ I_n $, определяющее замыкание произведения $ n $ коммутаторов. 
Очевидно, что $ I_1 \supseteq I_2 \supseteq I_3 \supseteq \ldots $.

Коммутаторная подгруппа в $ S $ - объединение произведений из $ n $ 
коммутаторов по всем $ n $, поэтому идеалом функций, зануляющихся на $ S $ является $ I = \bigcap I_n $. Замкнутое множество, определяемое идеалом $ I $,
является замыканием коммутаторной подгруппы. Это замкнутая нормальная подгруппа в 
$ S $, которую будем называть коммутантом $ \dg S $.
Итерируя эту процедуру, получаем цепочку замкнутых подгрупп $ \dg^{n} S $. Если $ S $ разрешима как абстрактрая группа, 
то последовательность $ \dg^{n} S $ достигает $ \{e\} $.


Все эти рассуждения могут быть проведены и в общем случае. Пусть $ G $ - аффинная групповая суперсхема над полем $ K $. Имеем отображения $ G^{2n} \to G $, которые соответствут $ K[G] ~\to ~\otimes^{2n} ~K[G] $ с ядрами $ I_n $, удовлетворяющими условию $ I_1 \supseteq I_2 \supseteq \ldots $. Если $ f \in I_{2n} $, 
то $ \Delta(f) $ обращается в нуль на $ K[G] / I_n \otimes K[G] / I_n $ в силу того, что при перемножении двух произведений по $ n $ коммутаторов образуется произведение $ 2n $ коммутаторов. Поэтому $ I = \bigcap I_n $ определяет замкнутую подгруппу $ \dg S $.


Будем называть супергруппу $ G $ \defn{разрешимой}, если $ \dg^{n} G $ тривиальна для некоторого $ n $.


\remark {
Все коммутаторы $ G(R) $ лежат в $ \dg G(R) $, $ \dg G $ - нормальная подгруппа в $ G $.
}

\theorem {
Пусть $ G $ -- алгебраическая супергруппа. Если $ G $ связна, то и $ \dg G $ связна.
}
\proof {
\qedhere
}

\proposition {
$ I = \bigcap I_n $ -- суперидеал Хопфа
}
\proposition {
$ \dg G $ -- нормальная подгруппа в $ G $.
}
\proposition {
$ I_{n+1} \subseteq I_{n} $
}
\proposition {
$ I $ -- наименьшая замкнутая подгруппа $ G $, содержащая произведение любых коммутаторов
}
\proposition {
$ G $ абелева $ \iff Lie(G) $ абелева.
}
\proof {
Достаточно доказать, что $ Dist(G) $ абелева $ \iff K[G]^{*} $ кокоммутативна.
}
