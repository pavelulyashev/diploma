\begin{subsection}{Супералгебры и супермодули}\label{superalgebras}
  Следуя~\cite{kleshchev} и ~\cite{some_properties_supergroups},
  приведем накоторые стандартные определения и теоремы.
  \newline

  Везде далее $ \K $ --- алгебраически замкнутое поле характеристики
  $ p $~(возможно, $ p = 0 $). Если $ p = 0 $, то предполагается, что $ p \neq 2 $.
  Супераналог произвольной алгебраической системы определяется введением
  $\Zz$-градуировки, относительно которой все структурные функции однородны.
  Так, супералгебра --- $\Zz$-градуированное пространство, такое что
  четность произведения двух $\Zz$-однородных элементов равна сумме их четностей
  по модулю 2. Если не оговорено противное, то морфизм двух суперсистем
  одинаковой сигнатуры сохраняет $\Zz$-градуировку. Подробнее с градуированными
  пространствами можно познакомиться в~\cite{arjantsev}.

  Приведем более формальные определения:
  \begin{definition}
    Будем называть (векторным) суперпространством пространство $ V = V_0 \oplus V_1 $
    над полем $ \K $. Если $ \dim V_0 = m, \dim V_1 = n $, то $ \dim V = m + n,
    \sdim V = (m, n) $. Элементы из $ V_0 $ называются четными, из $ V_1 $ --- нечетными.
  \end{definition}
  \begin{definition}
    Супералнеброй над полем $ \K $ называется суперпространство 
    $ A = A_0 \oplus A_1 $, наделенное структурой унитарной ассоциативной
    $\K$-алгебры, такое что $ ~A_i A_j \subset A_{i+j}, \quad \text{где} i, j = 0, 1. $
  \end{definition}
  Под \defn{суперидеалом} $ A $ подразумевается однородный идеал алгебры.

  Пусть $ V, W $ --- суперпространства. Их тензорное произведение
  наделяется структурой суперпространства по правилу $ |v \o w| = |v| + |w| ~(\mod 2) $,
  где прямыми скобками обозначена четность соответствующего элемента. Итерируя
  эту процедуру, можно определить тензорное произведение любого числа суперпространств.

  Для произвольных суперпространств $ V, W $ пространство $ \Hom_{\K}(V, W) $
  наделяется стандартной структурой суперпространства по правилу
  $ \p \in \Hom_{\K}(V, W)_i,~i = 0, 1 $, если $ \p(V_s) \subseteq W_k $, где
  $ i + s \equiv K~(\mod 2) $. В частности, если определить на $ \K $ структуру
  суперпространства с $ \K_0 = \K, \;\K_1 = 0 $, тогда
  $ V^* = \Hom_{\K}(V, \K) = V_0^* \oplus V_1^* $.

  \begin{definition}
    Пусть $ A $ --- супералгебра. (Левым) $A$-супермодулем называется
    суперпространство $ V $, которое является $ A $-модулем в обычном смысле,
    такое что $ A_i V_j \subseteq V_{i+j} $ для $ i, j \in \Zz $.
    Правый супермодули определяются аналогично.
  \end{definition}
  Под гомоморфизмом $ f: V \to W $ левых $A$-супермодулей подразумевается
  линейной отображение (не обязательно однородное), такое что
  $$ f(av) = \sgn{f}{a} a f(v), \qquad a \in A, ~v \in V, $$
  а для правых $A$-супермодулей
  $$ f(va) = f(v) a, \qquad a \in A, ~v \in V. $$

  Пусть $ A, B $ --- супералгебры, а $ V, W $ --- (левые) супермодули над
  $ А $ и $ B $ соответственно. Тогда тензорное произведение $ A \o B $ имеет
  структуру супералгебры относительно умножения
  $ a \o b \,\cdot \,c \o d = \sgn{b}{c} ac \o bd, ~a,c \in A, ~b,d \in B $.
  Более того, суперпространство $ V \o W $ будет
  $ A \o B $-супермодулем относительно действия
  $ a \o b \,\cdot \,v \o w = \sgn{b}{v} ac \o bd, ~a, \in A, ~b \in B, ~v \in V, ~w \in W $.

  Супералгебра $ A $ называется \defn{коммутативной}, если для любых однородных
  $ a, c \in A $ выполняется $ ac = \sgn{a}{c} ca $. Несложно убедиться, что
  если супералгебры $ A $ и $ B $ коммутативны, то супералгебра $ A \o B $ также
  коммутативна.
\end{subsection}

\begin{subsection}{$\K$-функторы}
  Определения, данные в ~\cite{jantzen} для обычного случая, можно почти дословно
  перенести на ~суперслучай. Некоторые из ~них можно найти в ~\cite{affine_quotients}.

  Введем некоторые предварительные обозначения. $ \K $ -- произвольное поле,
  $ \SAlg $ --- категория супералгебр над ~полем $ \K $,
  $ \Sets $ --- категория множеств, $ \Groups $ --- категория групп.

  \begin{definition}
    $\K$-функтором назовем функтор из~категории $ \SAlg $ в~$ \Sets $.
  \end{definition}

  Для $\K$-функторов $ X, X' $ обозначим через $ \Mor(X, X') $ множество морфизмов из $ Х $ в $ X' $.

  \begin{definition}
    Пусть $ X $ --- $\K$-функтор. $\K$-функтор $ Y $ называется
    подфунктором функтора $ X $, если $ ~\forall ~A, A' \in \SAlg
    \;~\forall ~\p \in \HomSAlg(A, A') $ выполнены условия:
    $ Y(A) \subset X(A) $ и $ Y(\p) = X(\p)|_{Y(A)} $.
  \end{definition}

  Для любого семейства подфункторов $ \{Y_i\}_{i \in I} \subset X $ определим
  функтор-пересечение $ \bigcap_{\substack{i \in I}} Y_i $ следующим образом:
  $$ ( \bigcap_{\substack{i \in I}} Y_i )(A) = \bigcap_{\substack{i \in I}} Y_i (A). $$

  Для ~$ f \in \Mor(X, X') ~\forall ~Y' \subseteq X' $ определим функтор-прообраз
  $$ (f^{-1}(Y'))(A) = f(A)^{-1} (Y'(A)) \qquad \text{для} ~A \in \SAlg. $$
  %
  Нетрудно убедиться, что $ \bigcap_{i \in I}Y_i $ и $ f^{-1}(Y') $ ---
  подфункторы $ X $.
  %
  \begin{definition}
    Прямым произведением $\K$-функторов $ X_1$ и $ X_2 $ называется функтор
    $ (X_1 \x X_2)(A) = X_1(A) \x X_2(A) ~\;\text{для} ~A \in \SAlg $.
  \end{definition}

  Проекции $ p_i: X_1 \x X_2 \to X_i $ являются морфизмами функторов, и
  $ (X_1 \x X_2, p_1, p_2) $ обладает обычными свойствами прямого произведения.
\end{subsection}

