\begin{lemma}\label{Lie(Gev) = L_0}
  Обозначим $ \Lie(G) = L = L_0 \oplus L_1 $. $ \Lie(G_{ev}) = L_0 $.
%  \proof {
%    \qedhere
%  }
\end{lemma}

\begin{lemma} \label{abelian G and Lie(G)}
  Аффинная групповая суперсхема $ G $ абелева $ \iff $ $ \Lie(G) $ абелева.
%  \proof {
%    Достаточно доказать, что $ \Dist(G) $ абелева $ \iff \K[G]^{*} $ кокоммутативна.
%    \qedhere
%  }
\end{lemma}

\begin{theorem}[Кац] \label{kac}
  Супералгебра Ли $ L = L_0 \oplus L_1 $ разрешима тогда и только тогда,
  когда разрешима алгебра Ли $ L_0 $.
\end{theorem}
Доказательство можно найти в статье \cite{kac}. Теперь все готово
для доказательства основной теоремы этой работы.

\begin{theorem}
  Пусть $ \charc K = 0 $, $ G $ - связная алгебраическая аффинная групповая суперсхема.
  $ G $ разрешима $ \iff $ $ \Lie(G)$ разрешима $ \iff $ $ G_{ev} $ разрешима.

  \proof{
    \begin{trivlist}
      \item а)
        Предположим, что $ G $ разрешима, т.е. для некоторого $ n \in \N $
        \begin{equation}\label{g condition}
          G ~\rhd ~G' ~\rhd ~G'' ~\rhd ~\ldots ~\rhd ~G^{(n)} = 1.
        \end{equation}
        % Exact sequence for first derived subgroup
        Рассмотрим абелев фактор $ G/G' $. Отображение $ \pi: G \to G/G' $ является
        эпиморфизмом алгебраических аффинных групповых суперсхем,
        следовательно по утверждению~\ref{exact sequence for Lie} имеем точную последовательность
        \begin{align*}
          0 \to \Lie(G') \to \Lie(G) \to \Lie(G/G') \to 0,
        \end{align*}
        Откуда получаем, что $ \Lie(G') \subseteq \Lie(G) $. Поскольку фактор
        $ G/G' $ абелев, то по лемме~\ref{abelian G and Lie(G)} $ \Lie(G/G') $ абелева,
        а значит и фактор $ \Lie(G) / \Lie(G') = \Lie(G/G') $ абелев.
        Рассматривая таким образом все факторы цепочки~(\ref{g condition}), получаем цепочку
        \begin{align*}
          \Lie(G) ~\rhd ~\Lie(G') ~\rhd ~\Lie(G'') ~\rhd ~\ldots ~\rhd ~\Lie(G^{(n-1)} ~\rhd 0,
        \end{align*}
        в которой все факторы абелевы, то есть $ \Lie(G) $ разрешима.
        \newline

        Обратно, предположим, что $ \Lie(G) $ разрешима, то есть существует цепочка
        \begin{equation} \label{lie condition}
          \Lie(G) ~\rhd ~I_1 ~\rhd ~I_2 ~\rhd ~\ldots ~\rhd ~I_n = 0.
        \end{equation}
        Рассмотрим $ I_{n-1} $. Т.к. $ I_n = 0 $,  то $ I_{n-1} $ ---
        максимальный суперидеал $ \Lie(G) $. При этом он абелев, как и все суперидеалы этой цепочки.
        Тогда по теореме~\ref{Exists H in G: Lie(H) = I} существует нормальная суперподсхема
        $ H_{n-1} \lhd G: \Lie(H_{n-1}) = I_{n-1} $. Суперидеал $ I_{n-1} $ абелев $ ~\hence $
        ~$ \Lie(H_{n-1}) $ абелева $ ~\hence $ абелевы ~$ H_{n-1} $ и $ G/H_{n-1} $.

        Теперь рассмотрим $ \Lie(G) / I_{n-1} = \Lie(G/H_{n-1}) $. $ I_{n-1} $ ---
        максимальный абелев суперидеал $ \Lie(G/H_{n-2}) ~\hence
        ~\exists ~H_{n-2} ~\lhd ~G/H_{n-1}: ~\Lie(H_{n-2}) = I_{n-2} $.
        Таким образом, мы получили $ \Lie(H_{n-1}) \lhd \Lie(H_{n-2}) $,
        а по лемме~\ref{subgroups iff Lie subalgebras} получаем, что $ H_{n-1} \lhd H_{n-2} $.
        %
        Аналогичным образом продолжая разбирать цепочку~(\ref{lie condition}) вверх, получаем цепочку
        \begin{align*}
          G ~\rhd ~H_1 ~\rhd ~\ldots ~\rhd ~H_{n-2} ~\rhd ~H_{n-1} ~\rhd ~E,
        \end{align*}
        в которой все факторы абелевы, то есть $ G $ разрешима.

        Таким образом, мы доказали, что $ G $ разрешима $ \iff $ $ \Lie(G)$ разрешима.

      \item б) По теореме~\ref{kac} $ \Lie(G) $ разрешима тогда и только тогда,
        когда разрешима $ \Lie(G)_0 $. По лемме~\ref{Lie(Gev) = L_0} $ \Lie(G)_0 = \Lie(G_{ev}) $,
        а из первой части доказательства следует, что $ \Lie(G_{ev}) $ разрешима
        тогда и только тогда, когда $ G_{ev} $ разрешима.
        Таким образом доказана вторая эквивалентность, а с ней и вся теорема.
        \qedhere
    \end{trivlist}
  }
\end{theorem}
