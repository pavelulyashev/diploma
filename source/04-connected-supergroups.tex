В этом разделе будут введены понятия связности~(\cite{waterhouse}) и
псевдосвязности~(\cite{affine_quotients}) аффинной групповой суперсхемы,
а также указана их эквивалентность для алгебраической аффинной
групповой суперсхемы над полем характеристики 0. Доказательства
сопутствующих фактов в основном опущены, поскольку опираются на понятия,
которые мы не будем вводить в этой работе.

\begin{subsection}{Топология Зарисского и связность аффинных груповых суперсхем}
  Сначала рассмотрим понятие связной компоненты для обычного случая, который
  подробно описан в \cite{waterhouse}, главы 5-6, 
  а затем перенесем полученные понятия на суперслучай.
  \newline

  Пусть $ A $ --- коммутативная алгебра. Множество
  $ \{ p ~| ~p \text{ --- простой идеал алгебры } A \} = \Spec A $
  называется \defn{спектром} алгебры $ A $. Подмножество спектра
  называется замкнутым, если оно имеет вид $ Z(I) = \{ p \in \Spec A ~| ~I \subseteq p \} $.
  для некоторого идеала $ I $. Соотношения $ \bigcap Z(I_n) = Z \left( \bigcap I_n \right),
  ~Z(I) \cup Z(J) = Z(IJ) $ определяют \defn{топологию Зарисского} на $ \Spec A $.

  Аффинную групповую схему $ X $ будем называть \defn{связной}, если
  $ \Spec\K[G] $ связен как топологическое пространство. По теореме из
  пункта~6.6~из~\cite{waterhouse} $ X $ связна $ \iff $ $ \Spec A $ неприводим,
  т.е. в $ \K[G] $ нет нетривиальных идемпотентов.
  \newline

  Теперь перейдем к суперслучаю. Пусть $ A $ --- коммутативная супералгебра.
  Рассмотрим \defn{суперспектр} $ \SSpec A = \{ p ~| ~p \text{ --- простой суперидеал } A \} $.

  Покажем, что в силу коммутативности $ A $ простой идеал $ A $ является суперидеалом.
  Пусть $ x = x_0 + x_1 \in p $. В силу коммутативности $ x_1^2 = \sgn{x_1}{x_1} x_1^2
  ~\hence ~x_1^2 = 0 \in p ~\hence x_1 \in p ~\hence x_0 \in p $.
  Получаем, что все нечетные элементы лежат в $ p $, поэтому $ p = p_0 \oplus A_1 $,
  где $ p_0 $ --- простой идеал $ A_0 $.

  Таким образом, $ \SSpec A = \Spec A_0 $. Поскольку $ A_0 = A_0 / A_1^2 = A / A A_1 $,
  то $ \SSpec A = \SSpec (A / A A_1) $. Аналогично определяем топологию Зарисского:
  замкнутое подмножество $ Z(I) = \{ p \in \SSpec A ~| ~I \subseteq p \} $
  определяется суперидеалом $ I $. В силу вышеизложенного
  топология Зарисского на $ \SSpec A $ равна топологии Зарисского на $ \Spec A $.

  Можно доказать, что $ A A_1 $ --- суперидеал Хопфа.
  $ G_{ev} = V(A A_1) = \SSp (A / A A_1) $ --- наибольшая четная суперподсхема $ G $.

  Для нас будет важно, что
  \begin{proposition}
    Аффинная групповая суперсхема $ G $ связна тогда и только тогда, когда
    $ G_{ev} $ связна.
  \end{proposition}

\end{subsection}

\begin{subsection}{Псевдосвязная компонента}
  \begin{definition}
    Алгебраическая аффинная групповая суперсхема $ G = \SSp A $ называется
    псевдосвязной, если $ \bigcap_{n \geqslant 0} \M^n = 0 $, где $ \M = \ker\mci_G $.
  \end{definition}

  \begin{lemma}
    Пусть $ G $ --- алгебраическая аффинная групповая суперсхема. Суперидеал
    $ I = \bigcap_{n \geqslant 0} \M^n $ является суперидеалом Хопфа, а
    аффинная групповая суперподсхема $ G^{[0]} = V(I) $ нормальна и связна.
    \begin{comment}
    \proof {
      По определению $ \mci_G(\M) = \M $.
      $$
        \mcm_G(\M^n) \subseteq \sum_{0 \leqslant i \leqslant} \M^i \o \M^{n-i}
        \subseteq \bigcap_{0 \leqslant i \leqslant} (\M^i \o A + A \o \M^{n-i})
      $$
      $$
        \mcm_G(I) \subseteq \bigcap_{n \geqslant 0} \mcm_A(\M^n) \subseteq
        \bigcap_{n \geqslant 0} (\M^n \o A + A \o \M^n) = I \o A + A \o I.
      $$
      \qedhere
    }
    \end{comment}
  \end{lemma}

  $ G^{[0]} $ называется \defn{псевдосвязной компонентой} $ G $. Очевидно, что
  $ G $ псевдосвязна тогда и только тогда, когда $ G = G^{[0]} $.

  \begin{lemma}[Теорема Крулля о пересечении]
    Пусть $ A $ --- конечнопорожденная коммутативная супералгебра и $ V $ ---
    конечнопорожденный $ A $-суперкомодуль. Для любого суперидела $ I \subseteq A $
    $ \bigcap_{n \geqslant 0} I^n V = {v \in V ~| ~\exists ~x \in I_0 : (1 - x)v = 0} $.
  \end{lemma}

  \begin{proposition}\label{exact sequence for Lie}
    Пусть $ \pi : G \to H $ --- эпиморфизм алгебраических аффинных групповых суперсхем.
    Если $ \charc\K = 0 $, то порожденная эпиморфизмом короткая последовательность
    супералгебр Ли
    $$ 0 \to \Lie(\ker\pi) \to \Lie(G) \xrightarrow{d\pi} \Lie(H) \to 0 $$
    является точной.
  \end{proposition}

  \begin{lemma}\label{subgroups iff Lie subalgebras}
    Пусть $ G $ --- алгебраическая аффинная груповая суперсхема, $ H_1, H_2 $ ---
    cуперподсхемы $ G $, $ H_1 $ псевдосвязна. Тогда
    $ H_1 \subseteq H_2 \iff \Dist(H_1) \subseteq \Dist(H_2) $, а если $ \charc\K = 0 $,
    то $ H_1 \subseteq H_2 \iff \Lie(H_1) \subseteq \Lie(H_2) $.
  \end{lemma}

  \begin{lemma}
    Если $ G $ псевдосвязна или связна, то из $ \Lie(G) = 0 $ следует $ G = E $.
    В частности, если $ \charc\K = 0 $ и $ G $ алгебраическая, то $ G^{(0)} = G^{[0]} $.
  \end{lemma}

  Это важное утверждение позволяет при рассмотрении алгебраических
  аффинных групповых суперсхем в случае поля нулевой характеристики
  использовать определения связности и псевдосвязности как эквивалентные.

\end{subsection}


\begin{subsection}{Соответствие нормальных суперподсхем $ G $
                   максимальным абелевым суперидеалам $ \Lie(G) $ }
  \begin{definition}
    Подфунктор $ \Z(G) $ групового $ \K $-функтора $ G $ называется центральным,
    если $ H $ -- подфунктор в $ G $ и $ ~\forall A \in \SAlg \; H(A) $ ---
    центральная подгруппа в $ G(A) $.
  \end{definition}

  \begin{proposition} \label{Z(G) closed in G}
    Пусть $ G $ --- аффинная групповая суперсхема.
    $ \Z(G) $ --- замкнутая аффинная групповая суперподсхема в $ G $.
    \proof {
      Запишем формально определение центрального подфунктора:
      \begin{align*}
        \Z(G)(A) = \{ x \in G(A) ~|~ \forall B \in \SAlg ~\forall ~\p: A \to B
        \quad G(\p)(x) g = g G(\p)(x) \}.
      \end{align*}
      %
      Понятие центральной подгруппы подразумевает отображение
      $ [g, x] = g^{-1} x^{-1} g x $. Соответствующий ему коморфизм
      выглядит следующим образом:
      $$ \mnu(f) = \sum \sgn{f_2}{f_3} \mci(f_1) f_3 \o \mci(f_2) f_4 =: \sum u_1 \o u_2. $$
      Покажем, что суперидеал, порожденный элементом $ (u_2 - \mcu(u_2)) $,
      определяет $ \Z(G) $, т.е. $ \Z(G)(A) = \{ x \in G(A) ~|~ x(u_2 - \mcu(u_2)) = 0 \} $.

      \begin{trivlist}
        \item \quad a) Пусть $ x \in \Z(G)(A) $.
          Сначала заметим, что $ \forall ~g ~[g, x] = 1 $, откуда
          следует, что $ (g \o x)(f) = \mcu(f) $. Теперь рассмотрим
          \begin{align*}
            \sum u_1 \o \mcu(u_2) = \sum \sgn{f_2}{f_3} \mci(f_1) f_3 \o \mcu(\mci(f_2)) \mcu(f_4) = (*).
          \end{align*}
          Поскольку $ \mcu(f_4) $ --- скалярная величина, то ее можно перенести
          в первую часть тензора перед $ f_3 $ (в этом случае знак перед тензором не изменится).
          Из аксиомы коедницы и обозначений Свидлера получаем $ f_4 \mcu(f_3) = f_3 $.
          Учитявая соотношение~(\ref{antiendomorphism}), имеем
          \begin{align*}
            (*) = \sum \sgn{f_2}{f_3} \mci(f_1) f_3 \o \mcu(f_2).
          \end{align*}
          Снова используем аксиому коединицы (при этом $ f_2 $ перескакивает через $ f_3 $,
          поэтому $ \sgn{f_2}{f_3} $ исчезает), а затем аксиому антипода, получаем
          \begin{align*}
            (*) = \sum \mcu(f) \o 1.
          \end{align*}

          Взяв теперь $ B = A \o \K[G], ~g = id_B $, получаем
          $ (g \o x)(\sum u_1 \o \mcu(u_2)) = \mcu(f) $, откуда следует, что
          $ (g \o x)(\sum u_1 \o (u_2 - \mcu(u_2))) = 0 $, а в силу того, что
          $ g = id_B $, получаем $ x(u_2 - \mcu(u_2)) = 0 $.
        \item \quad б) Обратно, если $ x(u_2 - \mcu(u_2)) = 0 $, $ I $ --- суперидеал,
          порожденный элементом $ u_2 - \mcu(u_2) $, то $ x \in \Z(G)(A) $.
          \qedhere
      \end{trivlist}
    }
  \end{proposition}

  \begin{lemma} \label{Lie(Z(G)) = Z(Lie(G))}
    Если $ G $ связна, $ \charc\K = 0 $, то $ \Lie(\Z(G)) = \Z(\Lie(G)) $.
  \end{lemma}

  \begin{theorem} \label{Exists H in G: Lie(H) = I}
    Пусть $ \charc\K = 0 $, $ G $ --- связная аффинная групповая суперсхема, \\
    $ I $ --- максимальный абелев суперидеал в $ \Lie(G) $.
    Существует $ H \lhd G : \Lie(H) = I $.
    \proof {
      Обозначим $ L = \Lie(G) $.
      Доказательство проведем индукцией по $ \dim L $. База индукции очевидна:
      для $ I = 0 $ достаточно взять $ H = E $. Предположим,
      что если $ H $ --- связная аффинная групповая суперсхема и $ \dim \Lie(H) < \dim L $,
      то утверждение выполнено для $ H $.

      Рассмотрим действие $ \Ad: G \to \GL(I) $, $ \ker \Ad = R $.
      Пусть $ J = \Lie(R) = \{ x \in L | [x, I] = 0 \} $.
      Очевидно, $ I \subseteq J $.

      Если $ \dim J \leqslant \dim L $, то по предположению индукции утверждение выполнено для $ R $,
      т.е. $ \exists \; H \lhd R : \Lie(H) = I $. Поскольку $ H \lhd R $
      и $ R \lhd G $ как ядро $ \Ad $, то $ H \lhd G $, следовательно, утверждение выполнено для $ G $.

      Рассмотрим случай $ \dim J = \dim L $.
      % J = L
      Т.к. $ G $ алгебраическая, то $ \dim L < \infty \hence J = L $.
      % I ⊆ Z(L)
      Отсюда следует, что $ [L, I] = 0 $, а в силу определения центра $ I \subseteq \Z(L) $.
      % I = Z(L)
      По условию $ I $ --- максимальный суперидеал $ \hence $ $ I $ не может быть
      собственным подмножеством $ \hence $ $ I = \Z(L) $.
      % I = Lie(Z(G))
      По лемме \ref{Lie(Z(G)) = Z(Lie(G))} получаем, что $ I = \Lie(\Z(G)) $, а
      из утверждения~\ref{Z(G) closed in G} $ \Z(G) \lhd G $.
    \qedhere
    }
  \end{theorem}

\end{subsection}
