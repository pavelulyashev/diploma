\begin{subsection}{Нормальные аффинные групповые суперподсхемы}
  Повторим некоторые определения и утверждения из \cite{affine_quotients}, раздел 6.
  \newline

  \begin{definition}
    Групповой подфунктор $ H $ $\K$-функторa $ G $ называется нормальным
    если $ ~\forall ~A \in \SAlg ~H(A) $ --- нормальная подгруппа в $ G(A) $.
  \end{definition}

  Если $ G $ --- аффинная групповая суперсхема и $ H $ ---
  замкнутая суперподсхема, то $ H \lhd G $ тогда и только тогда, когда
  $ H $ удовлетворяет одному из следующих условий:
  \begin{align*}
    \mnu_r(f) = \sum \sgn{f_1}{f_2} f_2 \o f_1 \mci_G(f_3) ~\in ~I_H \o \K[G],
  \end{align*}
  или
  \begin{align*}
    \mnu_l(f) = \sum \sgn{f_1}{f_2} f_2 \o \mci_G(f_1) f_3 ~\in ~I_H \o \K[G],
  \end{align*}
  для любого $ f \in I_H $. Первое условие называется условием
  \defn{правой нормальности}, а второе --- условием \defn{левой нормальности}.
  Для аффинных групповых суперсхем эти условия эквивалентны.

  Морфизм супералгебр $ \mnu_l $ дуален морфизму суперсхем $ G \x G \to G $,
  который задается отображением $ \mcon: (g_1, g_2) \mapsto g_2^{-1} g_1 g_2 $ для
  $ g_1, g_2 \in G(A), ~A \in \SAlg $. Симметричное отображение
  $ (g_1, g_2) \mapsto g_2 g_1 g_2^{-1} $ задает дуальный морфизм $ \mnu_r $.

  \begin{lemma}
    Пусть $ G $ --- аффинная групповая суперсхема, $ H \lhd G $. Тогда
    $ \Lie(H) $ --- суперидеал Ли $ \Lie(G) $.
  \end{lemma}


\end{subsection}

\begin{subsection}{Куммутант аффинной групповой суперсхемы}
  Перед тем, как дать определение разрешимой аффинной групповой суперсхемы,
  сначала необходимо определить коммутаторную суперподсхему. Нижеизложенное
  повторяет описание коммутанта для обычных схем (\cite{waterhouse}, глава 10),
  но более подробно.

  Пусть $ G $ - аффинная групповая суперсхема над полем $ \K $.
  Определим морфизм функторов $ \mpi: G \x G \to G $, переводящее
  $ (g, h) $ в коммутатор $ g h g^{-1} h^{-1} $. Ему соответствует коморфизм
  $ \mpi^*: \K[G] \to \K[G] \o \K[G] $. Обозначим $ I_1 = \ker\mpi^* $.
  Как ядро гомоморфизма супералгебр Хопфа $ I_1 $ --- суперидеал Хопфа.
  Аналогично для $ n \in \N $ имеем морфизм
  $ \mpi_n: (g_1, h_1, \ldots, g_n, h_n) \mapsto g_1 h_1 g_1^{-1} h_1^{-1} \cdots g_n h_n g_n^{-1} h_n^{-1} $
  и коморфизм $ \mpi_n^*: \K[G] \to \K[G]^{\o 2n} $. $ I_n = \ker\mpi_n^* $.

  \begin{proposition}
    $ I_{n+1} \subseteq I_{n} $.
    \proof {
      Очевидно, что $ \ker\mpi_{n} \subseteq \ker\mpi_{n+1} $. Переходя к коморфизмам,
      получаем, что $ \ker\mpi_{n+1}^* \subseteq \ker\mpi_{n}^* $.
      \qedhere
    }
  \end{proposition}

  Обозначим $ I = \bigcap_{n \geqslant 0} I_n $.

  \begin{proposition}
    $ I = \bigcap I_n $ -- суперидеал Хопфа.
    \proof {
      Для любого $ n \in \N \quad I_n $ --- суперидеал Хопфа, то есть
      $$ I \subseteq \ker\mcu, \quad
        \mcm(I_n) \subseteq \K[G] \o I_n + I_n \o \K[G], \quad
        \mci(I_n) \subseteq I.
      $$
      Очевидно, что $ I \subseteq \ker\mcu $ и $ \mci(I) \subseteq I $.

      Рассмотрим $ I_{2n} $. ~$ \mpi_{2n} = \mm \circ (\mpi_n \x \mpi_n) $,
      поэтому можно записать дуальную коммутативную диаграмму:
      \begin{align*}
        \begin{diagram}
          \node{G^{2n} \x G^{2n}}
            \arrow{e,t}{\mpi_n \x \mpi_n}
            \arrow{se,b}{\mpi_{2n}}
          \node{G \x G}
            \arrow{s,r}{\mm} \\
          \node[2]{G}
        \end{diagram} \qquad
        \begin{diagram}
          \node{\K[G]^{\o 2n} \o \K[G]^{\o 2n}}
          \node{\K[G] \x \K[G]}
            \arrow{w,t}{\mpi_n^* \o \mpi_n^*} \\
          \node[2]{\K[G]}
            \arrow{nw,b}{\mpi_{2n}^*}
            \arrow{n,r}{\mcm}
        \end{diagram}
      \end{align*}
      Отсюда получаем, что $ \mpi_{2n}^* = (\mpi_n^* \x \mpi_n^*) \circ \mcm $.
      Если $ \mcm(\p) = 0 $, то $ (\mpi_n^* \x \mpi_n^*)(\mcm(\p)) = 0 $, поэтому
      \begin{equation}
        \mcm(I_{2n}) \subseteq \ker(\mpi_n^* \x \mpi_n^*) = I_n \o \K[G] + \K[G] \o I_n.
      \end{equation}
      $ I = \bigcap I_n = \bigcap I_{2n} ~\hence ~\mcm(I) = \mcm(\bigcap I_{2n})
      \subseteq \bigcap (I_n \o \K[G] + \K[G] \o I_n) =
      \bigcap I_n \o \K[G] + \K[G] \o \bigcap I_n  = I \o \K[G] + \K[G] \o I $,
      что и требовалось для суперидеала Хопфа.
      \qedhere
    }
  \end{proposition}

  \begin{proposition}
    Замкнутая суперподсхема $ V(I) $, определяемая суперидеалом Хопфа $ I $,
    является наименьшей замкнутой суперподсхемой, содержащей произведения
    любых коммутаторов.
    \begin{comment}
    \proof {
      Достаточно доказать, что $ ~\forall ~n \in \N \quad \cl{\Im\mpi_n} = V(I_n) $.
    }
    \end{comment}
  \end{proposition}
  Это утвердждение расносильно тому, что $ ~\forall ~n \in \N \quad \cl{\Im\mpi_n} = V(I_n) $.

  \begin{definition}
    Замкнутая подгруппа $ V(I) = G' $, определяемая суперидеалом Хопфа $ I $,
    называется коммутантом аффинной групповой суперсхемы $ G $.
  \end{definition}

  \begin{proposition}
    $ G' $ --- нормальная суперподсхема аффинной групповой суперсхемы $ G $.
    \proof {
      Докажем, что $ \mnu(I) \subseteq I \o \K[G] $. Рассмотрим отображение
      $ f = \mcon \circ (\mpi_n \x id): ((g_1, h_1, \ldots, g_n, h_n), g) \mapsto
      g^{-1} g_1 h_1 g_1^{-1} h_1^{-1} \cdots g_n h_n g_n^{-1} h_n^{-1} g $.

      Покажем, что $ g^{-1} \mpi(g_1, h_1) g = x y x^{-1} y^{-1} $ для некоторых $ x, y $.
      $$ g^{-1} g_1 h_1 g_1^{-1} h_1^{-1} g =
      (g^{-1} g_1 g)(g^{-1} h_1 g^{-1})(g g_1^{-1} g^{-1})(g h_1^{-1} g), $$
      поэтому для $ x = g^{-1} g_1 g, ~y = g^{-1} h_1 g^{-1} $ получаем требуемое.
      Аналогичные рассуждения можно провести для $ g^{-1} \mpi(g_1, h_1, \ldots, g_n, h_n) g $.

      Это означает, что $ V(\ker f^*) = \cl{\Im f} ~\subseteq ~\cl{\Im \mpi_n} = V(I_n) $.

      Запишем соответствующие коммутативные диаграммы:
      \begin{align*}
        \begin{diagram}
          \node{G^{2n} \x G}
            \arrow{e,t}{\mpi_n \x id}
            \arrow{se,b}{f}
          \node{G \x G}
            \arrow{s,r}{\mcon} \\
          \node[2]{G}
        \end{diagram} \qquad
        \begin{diagram}
          \node{\K[G]^{\o 2n} \o \K[G]}
          \node{\K[G] \x \K[G]}
            \arrow{w,t}{\mpi_n^* \o id} \\
          \node[2]{\K[G]}
            \arrow{nw,b}{f^*}
            \arrow{n,r}{\mnu_l}
        \end{diagram}
      \end{align*}

      Получаем $ f^* = (\mpi_n^* \o id) \circ \mnu_l $.

      $ I_n \subseteq \ker((\mpi_n^* \o id) \circ \mnu_l) ~\hence
      (\mpi_n^* \o id)(~\mnu_l(I_n)) = 0 $, откуда получаем
      $$ \mnu_l(I_n) \subseteq \ker(\mpi_n^* \o id) =
      I_n \o \K[G] + \K[G] \o \ker id = I_n \o \K[G]. $$
      Следовательно, $ \mnu_l \left( \bigcap I_n \right) \subseteq
      \bigcap (I_n \o \K[G]) = (\bigcap I_n) \o \K[G] = I \o \K[G] $.
      \qedhere
    }
  \end{proposition}
  ~
  \\

  Стандартным образом определим $n$-й коммутант $ G^{(n)} $.

  \begin{definition}
    Будем называть аффинную групповую суперсхему $ G $ разрешимой,
    если $ G^{(n)} $ тривиальна для некоторого $ n $.
  \end{definition}

  Следующее утверждение будет важно для дальнейших рассуждений.
  Доказательство можно найти в \cite{waterhouse}.
  \begin{theorem}
    Пусть $ G $ алгебраическая. Если $ G $ связна, то и $ G' $ связна.
  \end{theorem}

\end{subsection}
