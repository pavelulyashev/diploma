Для того, чтобы сформулировать определение разрешимой супергруппы,
сначала необходимо определить коммутант супергруппы.

Пусть $ S $ --- алгебраическая матричная супергруппа. Рассмотрим отображение
$ S ~\times ~S ~\to ~S $, переводящее $ (x, y) $ в $ xyx^{-1}y^{-1} $.
Ядро $ I_{1} $ соотвествующего отображения $ \K[S] ~\to ~\K[S] ~\otimes ~\K[S] $
состоит из функций, зануляющихся на всех коммутаторах из $ S $;
таким образом, замкнутое множество, им определяемое, является замыканием коммутаторов.
Аналогично имеем отображение $ S^{2n} \to S $, переводящее $ (x_1, y_1, \ldots, x_n, y_n) $ в 
$ x_1y_1x_1^{-1}y_1^{-1} \cdots x_ny_nx_n^{-1}y_n^{-1} $.
Соответствующее отображение $ \K[S] ~\to ~\otimes^{2n} ~\K[S] $ имеет ядро
$ I_n $, определяющее замыкание произведения $ n $ коммутаторов. 
Очевидно, что $ I_1 \supseteq I_2 \supseteq I_3 \supseteq \ldots $.

Коммутаторная подгруппа в $ S $ --- объединение произведений из $ n $ 
коммутаторов по всем $ n $, поэтому идеалом функций, зануляющихся на $ S $ является $ I = \bigcap I_n $.
Замкнутое множество, определяемое идеалом $ I $, является замыканием коммутаторной подгруппы.
Это замкнутая нормальная подгруппа в $ S $, которую будем называть коммутантом $ \D S $.
Итерируя эту процедуру, получаем цепочку замкнутых подгрупп $ \D^{n} S $.
Если $ S $ разрешима как абстрактрая группа, то последовательность $ \D^{n} S $ достигает $ \{e\} $.


Все эти рассуждения могут быть проведены и в общем случае.
Пусть $ G $ - аффинная групповая суперсхема над полем $ \K $.
Имеем отображения $ G^{2n} \to G $, которые соответствут $ \K[G] ~\to ~\otimes^{2n} ~\K[G] $
с ядрами $ I_n $, удовлетворяющими условию $ I_1 \supseteq I_2 \supseteq \ldots $.
Если $ f \in I_{2n} $, то $ \Delta(f) $ обращается в нуль на $ \K[G] / I_n \otimes \K[G] / I_n $
в силу того, что при перемножении двух произведений по $ n $ коммутаторов образуется произведение $ 2n $ коммутаторов.
Поэтому $ I = \bigcap I_n $ определяет замкнутую подгруппу $ \D S $.

\begin{definition}
  Будем называть супергруппу $ G $ \defn{разрешимой}, если $ \D^{n} G $ тривиальна для некоторого $ n $.
\end{definition}


\begin{remark}
  Все коммутаторы $ G(R) $ лежат в $ \D G(R) $, $ \D G $ - нормальная подгруппа в $ G $.
\end{remark}

\begin{theorem}
  Пусть $ G $ -- алгебраическая супергруппа. Если $ G $ связна, то и $ \D G $ связна.
  \proof {
    \qedhere
  }
\end{theorem}


\begin{proposition}
  $ I = \bigcap I_n $ -- суперидеал Хопфа
\end{proposition}

\begin{proposition}
  $ \D G $ -- нормальная подгруппа в $ G $.
\end{proposition}

\begin{proposition}
  $ I_{n+1} \subseteq I_{n} $
\end{proposition}

\begin{proposition}
  $ I $ -- наименьшая замкнутая подгруппа $ G $, содержащая произведение любых коммутаторов
\end{proposition}

\begin{proposition}
  $ G $ абелева $ \iff Lie(G) $ абелева.

\end{proposition}
