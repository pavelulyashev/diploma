Перед тем, как дать определение разрешимой аффинной групповой суперсхемы,
сначала необходимо определить коммутаторную суперподсхему. Нижеизложенное
повторяет описание коммутанта для обычных схем (\cite{waterhouse}, глава 10),
но более подробно.

Пусть $ G $ - аффинная групповая суперсхема над полем $ \K $.
Определим морфизм функторов $ \pi: G \x G \to G $, переводящее
$ (g, h) $ в коммутатор $ g h g^{-1} h^{-1} $. Ему соответствует коморфизм
$ \pi^*: \K[G] \to \K[G] \o \K[G] $. Обозначим $ I_1 = \ker\pi^* $.
Как ядро гомоморфизма супералгебр Хопфа $ I_1 $ --- суперидеал Хопфа.
Аналогично для $ n \in \N $ имеем морфизм
$ \pi_n: (g_1, h_1, \ldots, g_n, h_n) \mapsto g_1 h_1 g_1^{-1} h_1^{-1} \cdots g_n h_n g_n^{-1} h_n^{-1} $
и коморфизм $ \pi_n^*: \K[G] \to \K[G]^{\o 2n} $. $ I_n = \ker\pi_n^* $.

\begin{proposition}
  $ I_{n+1} \subseteq I_{n} $.
\end{proposition}

Обозначим $ I = \bigcap_{n \geqslant 0} I_n $.

\begin{proposition}
  $ I = \bigcap I_n $ -- суперидеал Хопфа.
  \proof {
    Для любого $ n \in \N \quad I_n $ --- суперидеал Хопфа, то есть
    $$ I \subseteq \ker\mcu, \quad
      \mcm(I_n) \subseteq \K[G] \o I_n + I_n \o \K[G], \quad
      \mci(I_n) \subseteq I.
    $$
    Очевидно, что $ I \subseteq \ker\mcu $ и $ \mci(I) \subseteq I $.

    Рассмотрим $ I_{2n} $. ~$ \pi_{2n} = \mm \circ (\pi_n \x \pi_n) $,
    поэтому можно записать дуальную коммутативную диаграмму:
    \begin{align*}
      \begin{diagram}
        \node{G^{2n} \x G^{2n}}
          \arrow{e,t}{\pi_n \x \pi_n}
          \arrow{se,b}{\pi_{2n}}
        \node{G \x G}
          \arrow{s,r}{\mm} \\
        \node[2]{G}
      \end{diagram} \qquad
      \begin{diagram}
        \node{\K[G]^{\o 2n} \o \K[G]^{\o 2n}}
        \node{\K[G] \x \K[G]}
          \arrow{w,t}{\pi_n^* \o \pi_n^*} \\
        \node[2]{\K[G]}
          \arrow{nw,b}{\pi_{2n}^*}
          \arrow{n,r}{\mcm}
      \end{diagram}
    \end{align*}
    Отсюда получаем, что $ \pi_{2n}^* = (\pi_n^* \x \pi_n^*) \circ \mcm $.
    Если $ \mcm(\p) = 0 $, то $ (\pi_n^* \x \pi_n^*)(\mcm(\p)) = 0 $, поэтому
    \begin{equation}
      \mcm(I_{2n}) \subseteq \ker(\pi_n^* \x \pi_n^*) = I_n \o \K[G] + \K[G] \o I_n.
    \end{equation}
    $ I = \bigcap I_n = \bigcap I_{2n} ~\hence ~\mcm(I) = \mcm(\bigcap I_{2n})
    \subseteq \bigcap (I_n \o \K[G] + \K[G] \o I_n) =
    \bigcap I_n \o \K[G] + \K[G] \o \bigcap I_n  = I \o \K[G] + \K[G] \o I $,
    что и требовалось для суперидеала Хопфа.
    \qedhere
  }
\end{proposition}

\begin{proposition}
  Замкнутая суперподсхема $ V(I) $, определяемая суперидеалом Хопфа $ I $,
  является наименьшей замкнутой суперподсхемой, содержащая произведения
  любых коммутаторов.
\end{proposition}

\begin{definition}
  Замкнутая подгруппа $ V(I) = G' $, определяемая суперидеалом Хопфа $ I $,
  называется коммутантом аффинной групповой суперсхемы $ G $.
\end{definition}

\begin{proposition}
  $ G' $ --- нормальная суперподсхема аффинной групповой суперсхемы $ G $.
\end{proposition}

Стандартным образом определим $n$-й коммутант $ G^{(n)} $.

\begin{definition}
  Будем называть аффинную групповую суперсхему $ G $ разрешимой,
  если $ G^{(n)} $ тривиальна для некоторого $ n $.
\end{definition}

\begin{theorem}
  Пусть $ G $ алгебраическая. Если $ G $ связна, то и $ G' $ связна.
\end{theorem}
