Главной задачей данной работы было изучение основ теории аффинных групповых
схем и обобщение некоторых результатов на суперслучай.
Аффинные схемы были введены А.~Гротендиком в 1950-х гг. при построении теории схем
как обобщение понятия аффинного и квазипроективного многообразий.
Одним из главных инструментов теории аффинных схем является теория категорий,
хотя изначально теория строилась без теории категорий, в чем можно убедиться,
изучая традиционную алгебраическую геометрию (\cite{shafarevich}).
Основные понятия теории категорий можно найти в \cite{category_introduction}
или в работе С.~Маклейна, одного из авторов теории категорий \cite{mclane}.

В литературе аффинные групповые суперсхемы часто для краткости
называют супергруппами. В данной работе я буду для ясности использовать полное название.

Основной задачей этой работы является аналог теоремы Каца
для разрешимых аффинных групповых суперсхем. В.~Г.~Кац в работе \cite{kac}
о супералгебрах Ли доказал, что супералгебра Ли разрешима тогда и только тогда,
когда разрешима ее четная часть. Супералгебры Ли тесно связаны с теоретической
физикой, а в теории аффинных схем появляются при изучении супералгебр распределений.
Еще один алгебраический объект, тесно связанный с физикой --- алгебра Хопфа.
Аналогочно тому, что категория аффинных групповых схем дуальна категории алгебр Хопфа
(\cite{waterhouse}), аффинные групповые суперсхемы дуальны супералгебрам Хопфа,
что позволяет развивать одну и ту же теорию либо в терминах суперсхем,
либо в терминах супералгебр Хопфа в зависимости от ситуации.

%
В первом разделе собраны необходимые предварительные сведения: 
понятия супералгебры и супермодуля над супералгеброй, $\K$-функторы как функторы
из категории супералгебр над полем $\K$ в категорию множеств.
Во втором разделе определяется основной объект исследований этой работы ---
аффинные групповые схемы. Затем определяется супералгебра Хопфа
как объект, дуальный аффинной групповой суперсхеме. Такой порядок подачи материала
обусловлен тем, что сначала обуславливается возникновение кообъектов,
и только затем приводятся формальные определения.


Третий раздел описывает супералгебры распределений аффинных групповых суперсхем
и их связь с супералгебрами Ли. Некоторые дополнительные сведения для суперслучая
можно найти в \cite{affine_quotients}, исходные понятия алгебр распределений
аффинных групповых схем можно найти в \cite{waterhouse}.
Вводится понятие функтора супералгебры Ли $ \LieFun(G) $.

В четвертом разделе вводятся понятия связной ($ G^{(0)}$) и псевдосвязной ($ G^{[0]}$)
компонент аффинной групповой суперсхемы $ G $, а также их эквивалентность
для случая алгебраических аффинных групповых суперсхем над полем характеристики 0.
Основной результат этого раздела --- теорема о том, что максимальному
абелеву суперидеалу $ I $ связной аффинной групповой суперсхемы соответствует
нормальная суперподсхема $ H $, такая что $ \Lie(H) = I $.

В пятом разделе понятие разрешимой аффинной групповой схемы
(\cite{waterhouse}, гл. 10) переносится на суперслучай, доказывается
обоснованность этой аналогии. Главным результатом является теорема о том, что
коммунант связной алгебраической аффинной групповой суперсхемы связен,
что будет затем использовано при доказательстве основной теоремы этой работы.

В заключительной части доказывается аналог теоремы Каца о разрешимости
аффинных групповых суперсхем.

Иногда в работе встречается понятие аффинной (групповой) схемы, не приведенной
в тексте работы. Все понятия для аффинных схем аналогичны соответствующим
понятиям для аффинных суперсхем, если супералгебры заменить на алгебры.