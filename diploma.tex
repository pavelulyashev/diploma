\documentclass[a4paper,12pt]{article}
\usepackage{cmap}
\usepackage[utf8]{inputenc}
\usepackage[russian]{babel}


\usepackage{amsthm}
\usepackage{amsmath}
\usepackage{mathrsfs}
\usepackage[cmtip,arrow]{xy}
\usepackage{pb-diagram,pb-xy}

\newtheorem{theorem}{Теорема}
\newtheorem{lemma}{Лемма}
\newtheorem{corollary}{Следствие}
\newtheorem{proposition}{Утверждение}
\newtheorem{definition}{Определение}
\newtheorem{remark}{Замечание}

\newcommand{\defn}[1] {\textit{#1}}               % Italize definition
\newcommand{\charc}{\mathop{\mathrm{char}}}       % Characteristic
\renewcommand{\mod}{\mathrm{mod}\,}               % Modulo
\newcommand{\sdim}{\mathrm{sdim\,}}               % Superdimension

\def \isom {\simeq}

% Some acronyms for convenience
\def \p {\varphi}
\def \q {\psi}
\def \r {\rho}
\def \e {\varepsilon}
\def \sgn#1#2 {(-1)^{|#1||#2|}}

\def \K {K}                   % Main field
\def \M {\mathcal{M}}         % module M
\def \Z {\mathrm{Z}}          % Center
\def \N {\mathbb{N}}          % Natural numbers
\def \Zz {\mathbb{Z}_2}       % Z_2
\def \A {\mathbf{A}}          % Affine space

\def \Sets {\mathbf{Sets}}    % Category of sets
\def \Groups {\mathbf{Gr}}    % Category of groups
\def \Alg {\mathbf{Alg}_\K}   % Category of algebras
\def \SAlg {\mathbf{SAlg}_\K} % Category of superalgebras
\def \Mor {\mathrm{Mor}}      % Set of morphisms
\def \Hom {\mathrm{Hom}}      % Set of homomorphisms
\def \HomSAlg {\mathrm{Hom_{\SAlg}}}   % Set of homomorphisms of superalgebras

\def \D {\mathscr{D}}         % Commutator
\def \Ad {\mathbf{Ad}}        % Adjoint
\def \ad {\mathbf{ad}}        % adjoint
\def \LieFun {\mathbf{Lie}}   % Lie superalgebra functor
\def \Lie {\mathrm{Lie}}      % Lie superalgebra
\def \Dist {\mathrm{Dist}}    % Superalgebra of distributions
\def \GL {\mathrm{GL}}        % General linear
\def \SL {\mathrm{SL}}        % Special linear
\def \Sp {Sp\,}               % Affine scheme
\def \SSp {SSp\,}             % Affine superscheme

\def \Spec {Spec\,}           % Spectrum
\def \SSpec {SSpec\,}         % Superspectrum

\def \DN#1 {#1[\e_0, \e_1]}   % Superalgebra of dual numbers

\def \iff {\Leftrightarrow}   % If and only if
\def \hence {\Rightarrow}     % Hence, Consequently

\def \mi {i}                  % inverse
\def \mu {1}                  % unit
\def \mm {m}                  % multiplication
\def \mcu {\varepsilon}       % Counit of Hopf algebra
\def \mcm {\Delta}            % Comultiplication of Hopf algebra
\def \mci {s}                 % Antipode of Hopf algebra
\def \mca {\tau}              % Coaction on comodule

\def \x {\times}              % Cortesian product
\def \o {\otimes}             % Tensor product

\usepackage{setspace} % межстрочные интервалы
\onehalfspacing

\usepackage{misccorr} % в заголовках появляется точка, но при ссылке на них ее нет 
\usepackage{indentfirst} % после заголовков ставится абзацный отступ

\usepackage{geometry} % Меняем поля страницы
\geometry{left=3cm}
\geometry{right=1.5cm}
\geometry{top=2cm}
\geometry{bottom=2cm}

\bibliographystyle{plain}

\begin{document}
  \begin{titlepage}
  \newpage

  \begin{center}
    \begin{bf}
      Омский государственный университет им.~Ф.М.~Достоевского\\
        \vspace{2mm}
      Институт математики и информационных технологий \\
        \vspace{2mm}
      Кафедра алгебры
    \end{bf}

    \vfill\vfill

    \begin{LARGE}
      \textbf{Аналог теоремы Каца для разрешимых аффинных групповых суперсхем} \\
    \end{LARGE}

    \begin{Large}
      Дипломная работа \\
    \end{Large}
    Специальность <<Прикладная математика и информатика>>
  \end{center}

  \vfill

  \hfill \parbox{7cm}{
        Выполнил: \\
        студент группы МПС-703-О \\
        Уляшев Павел Александрович \\
        %
        \underline{{}\hspace{5cm}{}} \\
        \vspace{-0.7mm}
        \textit{(подпись студента)}

        \vspace{8mm}

        Научный руководитель: \\
        д.ф.-м.н., профессор \\
        Зубков Александр Николаевич \\
        %
        \underline{{}\hspace{5cm}{}} \\
        \vspace{-0.7mm}
        \textit{(подпись руководителя)}
    }

  \vfill

  \begin{center}
    \large Омск 2012
  \end{center}
\end{titlepage}

  \setcounter{page}{2}

  \section*{Введение}
    Главной задачей данной работы было изучение основ теории аффинных групповых
схем и обобщение некоторых результатов на суперслучай.
Аффинные схемы были введены А.~Гротендиком в 1950-х гг. при построении теории схем
как обобщение понятия аффинного и квазипроективного многообразий.
Одним из главных инструментов теории аффинных схем является теория категорий,
хотя изначально теория строилась без теории категорий, в чем можно убедиться,
изучая традиционную алгебраическую геометрию (\cite{shafarevich}).
Основные понятия теории категорий можно найти в \cite{category_introduction}
или в работе С.~Маклейна, одного из авторов теории категорий \cite{mclane}.

В литературе аффинные групповые суперсхемы часто для краткости
называют супергруппами. В данной работе я буду для ясности использовать полное название.

Основной задачей этой работы является аналог теоремы Каца
для разрешимых аффинных групповых суперсхем. В.~Г.~Кац в работе \cite{kac}
о супералгебрах Ли доказал, что супералгебра Ли разрешима тогда и только тогда,
когда разрешима ее четная часть. Супералгебры Ли тесно связаны с теоретической
физикой, а в теории аффинных схем появляются при изучении супералгебр распределений.
Еще один алгебраический объект, тесно связанный с физикой --- алгебра Хопфа.
Аналогочно тому, что категория аффинных групповых схем дуальна категории алгебр Хопфа
(\cite{waterhouse}), аффинные групповые суперсхемы дуальны супералгебрам Хопфа,
что позволяет развивать одну и ту же теорию либо в терминах суперсхем,
либо в терминах супералгебр Хопфа в зависимости от ситуации.

%
В первом разделе собраны необходимые предварительные сведения: 
понятия супералгебры и супермодуля над супералгеброй, $\K$-функторы как функторы
из категории супералгебр над полем $\K$ в категорию множеств.
Во втором разделе определяется основной объект исследований этой работы ---
аффинные групповые схемы. Затем определяется супералгебра Хопфа
как объект, дуальный аффинной групповой суперсхеме. Такой порядок подачи материала
обусловлен тем, что сначала обуславливается возникновение кообъектов,
и только затем приводятся формальные определения.


Третий раздел описывает супералгебры распределений аффинных групповых суперсхем
и их связь с супералгебрами Ли. Некоторые дополнительные сведения для суперслучая
можно найти в \cite{affine_quotients}, исходные понятия алгебр распределений
аффинных групповых схем можно найти в \cite{waterhouse}.
Вводится понятие функтора супералгебры Ли $ \LieFun(G) $.

В четвертом разделе вводятся понятия связной ($ G^{(0)}$) и псевдосвязной ($ G^{[0]}$)
компонент аффинной групповой суперсхемы $ G $, а также их эквивалентность
для случая алгебраических аффинных групповых суперсхем над полем характеристики 0.
Основной результат этого раздела --- теорема о том, что максимальному
абелеву суперидеалу $ I $ связной аффинной групповой суперсхемы соответствует
нормальная суперподсхема $ H $, такая что $ \Lie(H) = I $.

В пятом разделе понятие разрешимой аффинной групповой схемы
(\cite{waterhouse}, гл. 10) переносится на суперслучай, доказывается
обоснованность этой аналогии. Главным результатом является теорема о том, что
коммунант связной алгебраической аффинной групповой суперсхемы связен,
что будет затем использовано при доказательстве основной теоремы этой работы.

В заключительной части доказывается аналог теоремы Каца о разрешимости
аффинных групповых суперсхем.

Иногда в работе встречается понятие аффинной (групповой) схемы, не приведенной
в тексте работы. Все понятия для аффинных схем аналогичны соответствующим
понятиям для аффинных суперсхем, если супералгебры заменить на алгебры.
  
  \newpage

  \tableofcontents

  \newpage
  \section{Предварительные сведения}
    \begin{subsection}{Супералгебры и супермодули}\label{superalgebras}
  Следуя~\cite{kleshchev} и ~\cite{some_properties_supergroups},
  приведем накоторые стандартные определения и теоремы.
  \newline

  Везде далее $ \K $ --- алгебраически замкнутое поле характеристики
  $ p $~(возможно, $ p = 0 $). Если $ p = 0 $, то предполагается, что $ p \neq 2 $.
  Супераналог произвольной алгебраической системы определяется введением
  $\Zz$-градуировки, относительно которой все структурные функции однородны.
  Так, супералгебра --- $\Zz$-градуированное пространство, такое что
  четность произведения двух $\Zz$-однородных элементов равна сумме их четностей
  по модулю 2. Если не оговорено противное, то морфизм двух суперсистем
  одинаковой сигнатуры сохраняет $\Zz$-градуировку. Подробнее с градуированными
  пространствами можно познакомиться в~\cite{arjantsev}.

  Приведем более формальные определения:
  \begin{definition}
    Будем называть (векторным) суперпространством пространство $ V = V_0 \oplus V_1 $
    над полем $ \K $. Если $ \dim V_0 = m, \dim V_1 = n $, то $ \dim V = m + n,
    \sdim V = (m, n) $. Элементы из $ V_0 $ называются четными, из $ V_1 $ --- нечетными.
  \end{definition}
  \begin{definition}
    Супералнеброй над полем $ \K $ называется суперпространство 
    $ A = A_0 \oplus A_1 $, наделенное структурой унитарной ассоциативной
    $\K$-алгебры, такое что $ ~A_i A_j \subset A_{i+j}, \quad \text{где} i, j = 0, 1. $
  \end{definition}
  Под \defn{суперидеалом} $ A $ подразумевается однородный идеал алгебры.

  Пусть $ V, W $ --- суперпространства. Их тензорное произведение
  наделяется структурой суперпространства по правилу $ |v \o w| = |v| + |w| ~(\mod 2) $,
  где прямыми скобками обозначена четность соответствующего элемента. Итерируя
  эту процедуру, можно определить тензорное произведение любого числа суперпространств.

  Для произвольных суперпространств $ V, W $ пространство $ \Hom_{\K}(V, W) $
  наделяется стандартной структурой суперпространства по правилу
  $ \p \in \Hom_{\K}(V, W)_i,~i = 0, 1 $, если $ \p(V_s) \subseteq W_k $, где
  $ i + s \equiv K~(\mod 2) $. В частности, если определить на $ \K $ структуру
  суперпространства с $ \K_0 = \K, \;\K_1 = 0 $, тогда
  $ V^* = \Hom_{\K}(V, \K) = V_0^* \oplus V_1^* $.

  \begin{definition}
    Пусть $ A $ --- супералгебра. (Левым) $A$-супермодулем называется
    суперпространство $ V $, которое является $ A $-модулем в обычном смысле,
    такое что $ A_i V_j \subseteq V_{i+j} $ для $ i, j \in \Zz $.
    Правый супермодули определяются аналогично.
  \end{definition}
  Под гомоморфизмом $ f: V \to W $ левых $A$-супермодулей подразумевается
  линейной отображение (не обязательно однородное), такое что
  $$ f(av) = \sgn{f}{a} a f(v), \qquad a \in A, ~v \in V, $$
  а для правых $A$-супермодулей
  $$ f(va) = f(v) a, \qquad a \in A, ~v \in V. $$

  Пусть $ A, B $ --- супералгебры, а $ V, W $ --- (левые) супермодули над
  $ А $ и $ B $ соответственно. Тогда тензорное произведение $ A \o B $ имеет
  структуру супералгебры относительно умножения
  $ a \o b \,\cdot \,c \o d = \sgn{b}{c} ac \o bd, ~a,c \in A, ~b,d \in B $.
  Более того, суперпространство $ V \o W $ будет
  $ A \o B $-супермодулем относительно действия
  $ a \o b \,\cdot \,v \o w = \sgn{b}{v} ac \o bd, ~a, \in A, ~b \in B, ~v \in V, ~w \in W $.

  Супералгебра $ A $ называется \defn{коммутативной}, если для любых однородных
  $ a, c \in A $ выполняется $ ac = \sgn{a}{c} ca $. Несложно убедиться, что
  если супералгебры $ A $ и $ B $ коммутативны, то супералгебра $ A \o B $ также
  коммутативна.
\end{subsection}

\begin{subsection}{$\K$-функторы}
  Определения, данные в ~\cite{jantzen} для обычного случая, можно почти дословно
  перенести на ~суперслучай. Некоторые из ~них можно найти в ~\cite{affine_quotients}.

  Введем некоторые предварительные обозначения. $ \K $ -- произвольное поле,
  $ \SAlg $ --- категория супералгебр над ~полем $ \K $,
  $ \Sets $ --- категория множеств, $ \Groups $ --- категория групп.

  \begin{definition}
    $\K$-функтором назовем функтор из~категории $ \SAlg $ в~$ \Sets $.
  \end{definition}

  Для $\K$-функторов $ X, X' $ обозначим через $ \Mor(X, X') $ множество морфизмов из $ Х $ в $ X' $.

  \begin{definition}
    Пусть $ X $ --- $\K$-функтор. $\K$-функтор $ Y $ называется
    подфунктором функтора $ X $, если $ ~\forall ~A, A' \in \SAlg
    \;~\forall ~\p \in \HomSAlg(A, A') $ выполнены условия:
    $ Y(A) \subset X(A) $ и $ Y(\p) = X(\p)|_{Y(A)} $.
  \end{definition}

  Для любого семейства подфункторов $ \{Y_i\}_{i \in I} \subset X $ определим
  функтор-пересечение $ \bigcap_{\substack{i \in I}} Y_i $ следующим образом:
  $$ ( \bigcap_{\substack{i \in I}} Y_i )(A) = \bigcap_{\substack{i \in I}} Y_i (A). $$

  Для ~$ f \in \Mor(X, X') ~\forall ~Y' \subseteq X' $ определим функтор-прообраз
  $$ (f^{-1}(Y'))(A) = f(A)^{-1} (Y'(A)) \qquad \text{для} ~A \in \SAlg. $$
  %
  Нетрудно убедиться, что $ \bigcap_{i \in I}Y_i $ и $ f^{-1}(Y') $ ---
  подфункторы $ X $.
  %
  \begin{definition}
    Прямым произведением $\K$-функторов $ X_1$ и $ X_2 $ называется функтор
    $ (X_1 \x X_2)(A) = X_1(A) \x X_2(A) ~\;\text{для} ~A \in \SAlg $.
  \end{definition}

  Проекции $ p_i: X_1 \x X_2 \to X_i $ являются морфизмами функторов, и
  $ (X_1 \x X_2, p_1, p_2) $ обладает обычными свойствами прямого произведения.
\end{subsection}



  %
  % TODO: замыкание подфунктора (нужно при доказательстве утверждения в разделе 5)
  \newpage
  \section{Аффинные групповые суперсхемы}
    \begin{subsection}{Аффинные суперсхемы}\label{closed subfunctors}
  \begin{definition}
    $\K$-функтор $ \SSp R $, определенный как
    $$ (\SSp R)(A) = \HomSAlg(R, A) \qquad \text{для} ~A \in \SAlg, $$
    называется аффинной суперсхемой. Супералгебра $ R \in \SAlg $ называется координатной
    супералгеброй суперсхемы $ \SSp R $. Если $ X = \SSp R $, то $ R $ обозначается $ \K[X] $.
  \end{definition}
  Пусть $ X_1, X_2 $ --- аффинные суперсхемы. Тогда
  \begin{equation}
    \K[X_1 \x X_2] = \K[X_1] \o \K[X_2].
  \end{equation}

  \begin{definition}
    Аффинная суперсхема $ \A^{m|n} = \SSp \K[t_1, \ldots, t_m | z_1, \ldots, z_n] $
    называется аффинным $(m|n)$-суперпространством.
  \end{definition}
  Очевидно, что $ \A^{m|n} (B) = B_0^m \oplus B_1^n ~\text{для} ~B \in \SAlg $.
  В частности, $ \A^{1|1}(B) = B $ для любой супералгебры $ B $.

  \begin{definition}
    Аффинная суперсхема $ X $ называется алгебраической, если \\
    $ \K[X] \simeq \K[t_1, \ldots, t_m | z_1, \ldots, z_n] \,/ \,I $ для некоторых
    $ m, n \in \N $ и конечнопорожденного суперидеала ~$ I $.
  \end{definition}
  \begin{definition}
    Аффинная суперсхема $ X $ называется редуцированной, если
    $ \K[X] $ не содержит нильпотентных элементов, отличных от 0.
  \end{definition}

  \begin{definition}
    Пусть $ X $ --- аффинная суперсхема, $ I $ --- суперидеал $ \K[X]$.
    Подфунктор функтора $ X $, определенный как
    \begin{align*}
      V(I)(A) & = \{ \p \in \HomSAlg(R, A) ~| ~\p(I) = 0 \} \\
              & \simeq \{ x \in (\SSp R)(A) ~| ~f(x) = 0 ~\forall ~f \in I \}
    \end{align*}
    называется замкнутым подфунктором, соответствующим суперидеалу ~$ I $.
  \end{definition}
  Отображение $ I \mapsto V(I) $ из множества суперидеалов $ \K[X] $ в множество
  подфункторов $ X $ инъективно. Более точно,
  \begin{proposition}
    Для двух суперидеалов $ I, I' $ супералгебры $ \K[X] $
    \begin{equation}
      I \subset I' \iff V(I) \supset V(I').
    \end{equation}
    \proof {
      Прямое утверждение тривиально, поэтому докажем верность обратного.
      Пусть $ V(I') \subset V(I)$. Рассмотрим каноническое отображение
      $ u: \K[X] \to \K[X] / I' $. $ u \in \HomSAlg(\K[X], \K[X] / I') = \SSp (\K[X] / I') $.
      Т.к. $ u(I') = 0 $, то $ u \in V(I')(\K[X] / I') $. Из условия $ V(I') \subset V(I) $
      следует, что $ u \in V(I')(\K[X] / I') \hence u(I) = 0 \hence I \subset I' $.
      \qedhere
    }
  \end{proposition}
  
  Замкнутый подфунктор является аффинной суперсхемой, т.к.
  $ V(I) \simeq \SSp (\K[X] / I) $.

  Замкнутые подфункторы определяют топологию на аффинной суперсхеме $ \SSp R $:
  \begin{align*}
    V(R) = \varnothing, & \quad
    V(0) = \SSp R, \\
    \bigcap_{j \in J} V(I_j) = V \left(\sum_{j \in J} I_j \right), & \quad
    \bigcup_{j \in J} V(I_j) = V \left(\prod_{j \in J} I_j \right)
  \end{align*}
  для любого семейства суперидеалов $ \{I_j\}_{j \in J} \subset R $.

  Пусть $ X_1, X_2 $ --- аффинные суперсхемы,
  $ I_1 \subset \K[X_1], ~I_2 \subset \K[X_2] $ --- суперидеалы.
  Несложно проверить, что
  \begin{equation}
    V(I_1) \x V(I_2) \simeq V(I_1 \o \K[X_2] + \K[X_1] \o I_2).
  \end{equation}


\end{subsection}

\begin{subsection}{Лемма Ионеды}
  \def \C {\mathcal{C}}   % Category

  Лемма Ионеды --- фундаментальное утверждение теории категорий --- позволяет
  вложить любую категорию $ \C $ в категорию функторов, определенных в $ \C $.
  В общем виде Лемму Ионеды можно найти в \cite{mclane}, в этой работе подробнее
  остановимся на случае для категории $ \SAlg $.
  
  \begin{lemma}[Ионеда] \label{yoneda}
    $ \forall ~R \in \SAlg ~\forall ~\K$-функторa $ X $ существует канонический изоморфизм
    $$ \Mor(\SSp R, X) \simeq X(R), $$
    который задается отображением $ f \mapsto f(R)(id_R) $.
    \proof {
      Пусть $ f \in \Mor(\SSp R, X) $. Сначала убедимся, что $ f(R)(id_R) \in X(R) $.
      Это следует из того, что $ f(R): (\SSp R)(R) = \HomSAlg(R, R) \to X(R) $.
      Далее убедимся, что приведенное отображение действительно является изоморфизмом.

      По определению морфизма функторов
      $ ~\forall ~A \in \SAlg \quad \forall ~u \in \HomSAlg(B, A) $ коммутативна диаграмма:

      \begin{equation}
        \begin{diagram}
          \node{(\SSp R)(B)}
            \arrow[2]{e,t}{f(B)}
            \arrow{s,l}{(\SSp R)(u)}
          \node[2]{X(B)}
            \arrow{s,r}{X(u)} \\
          \node{(\SSp R)(A)}
            \arrow[2]{e,t}{f(A)}
          \node[2]{X(A)}
        \end{diagram}
      \end{equation}

      Возьмем $ R $ в качестве $ B $ и получим, что
      $ f(A) \circ X(u) = (\SSp R)(u) \circ f(R) $. Обозначим $ x_f = f(R)(id_R) $.
      Принимая во внимание, что $ (\SSp R)(u)(id_R) = u \circ id_R $, получаем
      $$ f(A)(u) = X(u)(x_f). $$

      Отсюда видно, что $ f $ однозначно определяется $ x_f $. Осталось построить
      обратное отображение. Пусть $ x \in X(R) $ и $ A \in \SAlg $.
      Зададим $ f_x(A): \SSp R \to X(A) $ отображением $ u \mapsto X(u)(x) $.
      Несложно убедиться, что $ f_x \in \Mor(\SSp R, X) $
      и что $ x \mapsto f_x $ обратно отображению $ f \mapsto f_x $.
      \qedhere
    }

    \begin{corollary}\label{duality}
      Если взять $ X = \SSp R' $, то получим
      \begin{equation}
          \Mor(\SSp R,\SSp R') \isom \HomSAlg(R', R)
      \end{equation}
      для любых супералгебр $ R, R' $.
    \end{corollary}
      Обозначим эту биекцию $ f \mapsto f^* $ и
      будем называть $ f^* $ \defn{коморфизмом}, соответствующим $ f $.
      Таким образом, мы получили дуальность категорий
      аффинных суперсхем и супералгебр.

  \end{lemma}

\end{subsection}


\begin{subsection}{Групповые $\K$-функторы и аффинные групповые суперсхемы}
  \begin{definition}
    Групповым $\K$-функтором будем называть функтор из $ \SAlg $ в $ \Groups $.
  \end{definition}
  Если взять композицию группового функтора с забывающим функтором из $ \Groups $ в $ \Sets $,
  то групповой $\K$-функтор можно рассматривать как $\K$-функтор. Поэтому все результаты
  для $\K$-функторов можно перенести на групповые $\K$-функторы.

  Пусть $ G, H $ --- групповые $\K$-функторы. Обозначим через
  $ \Mor(G, H) $ множество морфизмов из $ G $ в $ H $, если рассматривать $ G $ и $ H $
  как $\K$-функторы;
  через $ \Hom(G, H) $ множество морфизмов групповых функторов.

  \begin{definition}
    Аффинная групповая суперсхема --- групповой $\K$-функтор,
    который является аффинной суперсхемой, если его рассматривать как функтор.
  \end{definition}

  \begin{definition}
    Пусть $ G $ --- групповой $\K$-функтор. $ H $ называется групповым подфунктором
    $ G $, если $ H $ --- подфунктор $ G $ и $ ~\forall ~A \in \SAlg ~H(A) $ ---
    подгруппа в $ G(A) $.
  \end{definition}

  Нетрудно убедиться, что пересечение групповых подфункторов --- групповой подфунктор,
  прообраз группового подфунктора относительно гомоморфизма --- групповой подфунктор.

  \begin{definition}
    Групповой подфунктор $ H $ функторa $ G $ называется нормальным
    (соответственно, центральным), если $ ~\forall ~A \in \SAlg ~H(A) $ --- нормальная
    (соответственно, центральная) подгруппа в $ G(A) $.
  \end{definition}

  \begin{definition}
    Пусть $ G $ --- аффинная групповая суперсхема. $ H $ --- замкнутая аффинная
    групповая суперподсхема, если $ H $ --- групповой подфунктор $ G $, который
    замкнут, если рассматривать $ H $ как подфунктор аффинной суперсхемы $ G $.
  \end{definition}

\end{subsection}


\begin{subsection}{Дуальность категорий аффинных групповых суперсхем и супералгебр Хопфа}
  Пусть $ G $ --- групповой $\K$-функтор, $ A, B \in \SAlg, ~u \in \HomSAlg(A, B) $,
  Групповая структура на $ G(A) $ определяет морфизмы $\K$-функторов
  (для каждого функтора коммутативная диаграмма из определения морфизма функторов):
  \begin{trivlist}
    \item умножение $ \mm_G: G \x G \to G \ (\mm_G (A) ~\text{--- умножение в группе } G(A)) $,
      \begin{equation}
         \begin{diagram}
          \node{G(A) \x G(A)}
            \arrow[2]{e,t}{\mm_G(A)}
            \arrow{s,l}{G(u) \x G(u)}
          \node[2]{G(A)}
            \arrow{s,r}{G(u)} \\
          \node{G(B) \x G(B)}
            \arrow[2]{e,t}{f(A)}
          \node[2]{G(B)}
        \end{diagram}
      \end{equation}

    \item единица $ \mu_G: \SSp\K \to G \ (\mu_G (A): f \mapsto 1_{G(A)} ~\text{для} ~f\in (\SSp\K)(A)) $,
      \begin{equation}
         \begin{diagram}
          \node{(\SSp\K)(A)}
            \arrow[2]{e,t}{\mu_G(A)}
            \arrow{s,l}{(\SSp\K)(u)}
          \node[2]{G(A)}
            \arrow{s,r}{(\SSp\K)(u)} \\
          \node{(\SSp\K)(B)}
            \arrow[2]{e,t}{\mu_G(A)}
          \node[2]{G(B)}
        \end{diagram}
      \end{equation}

    \item обратная функция $ \mi_G: G \to G \ (\mi_G(A): g \mapsto g^{-1} ~\text{для} ~g \in G(A)) $
      \begin{equation}
         \begin{diagram}
          \node{G(A)}
            \arrow[2]{e,t}{\mi_G(A)}
            \arrow{s,l}{G(u)}
          \node[2]{G(A)}
            \arrow{s,r}{G(u)} \\
          \node{G(B)}
            \arrow[2]{e,t}{\mi_G(A)}
          \node[2]{G(B)}
        \end{diagram}
      \end{equation}

  \end{trivlist}

  Пусть $ G $ --- аффинная групповая суперсхема. Согласно следствию~\ref{duality}
  каждому из этих морфизмов единственным образом советствует свой коморфизм.
  \begin{align*}
    \text{коумножение} & \quad \mcm_G = \mm_G^*: \K[G] \to \K[G] \o \K[G],\\
    \text{коединица} & \quad \mcu_G = \mu_G^*: \K[G] \to \K,\\
    \text{антипод} & \quad \mci_G = \mi_G^*: \K[G] \to \K[G],
  \end{align*}

  Из аксиом групповой структуры следуют аксиомы коумножения, коединицы и антипода.
  Ниже эти аксиомы записаны в виде коммутативных диаграмм.
  Ассоциативность умножения $ g_1(g_2 g_3) = (g_1 g_2) g_3 $
  переходит в коассоциативность коумножения:
  \begin{equation}\label{coassoc}
    \begin{diagram}
      \node{\K[G]}
        \arrow[2]{e,t}{\mcm}
        \arrow{s,l}{\mcm}
      \node[2]{\K[G] \o \K[G]}
        \arrow{s,r}{id \o \mcm}\\
      \node{\K[G] \o \K[G]}
        \arrow[2]{e,t}{\mcm \o 1}
      \node[2]{\K[G] \o \K[G] \o \K[G]}
    \end{diagram}
  \end{equation}
  %
  Аксиома единицы $ eg = ge = g $ переходит в аксиому коединицы:
  \begin{equation}\label{counit}
    \begin{diagram}
      \node{\K \o \K[G]}
        \arrow{e,=}
        \arrow{s,r}{id \o id}
      \node{\K[G]}
        \arrow{s,r}{\mcm \o id}
        \arrow{e,=}
      \node{\K[G] \o \K}
        \arrow{s,r}{id \o id} \\
      \node{\K \o \K[G]}
      \node{\K[G] \o \K[G]}
        \arrow{w,t}{\mcu \o 1}
        \arrow{e,t}{1 \o \mcu}
      \node{\K[G] \o \K}
    \end{diagram}
  \end{equation}
  %
  Аксиома обратного элемента $ g g^{-1} = g^{-1} g = e $ переходит в аксиому антипода:
  \begin{equation}\label{antipode}
    \begin{diagram}
      \node{\K[G] \o \K[G]}
        \arrow{e,t}{\mci \o id}
      \node{\K[G] \o \K[G]}
        \arrow{se,r}{\mm} \\
      \node{\K[G]}
        \arrow{n,l}{\mcm}
        \arrow{e,t}{\mcu}
        \arrow{s,l}{\mcm}
      \node{\K}
        \arrow{e,t}{\mu}
      \node{\K[G]}\\
      \node{\K[G] \o \K[G]}
        \arrow{e,t}{id \o \mci}
      \node{\K[G] \o \K[G]}
        \arrow{ne,r}{\mm}
    \end{diagram}
  \end{equation}

  Следуя Свидлеру, будем писать $ \mcm(c) = \sum c_1 \o c_2 $ 
  (подробнее о коструктурах и соответствующих обозначениях будет рассказано ниже).
  Тогда вышеуказанные аксиомы записываются в виде:
  \begin{align*}
    (\mcm \o id) \circ \mcm = \mcm \circ (id \o \mcm) & \qquad
    \sum c_{11} \o c_{12} \o c_2 = \sum c_1 \o c_{21} \o c_{22} =: \sum c_1 \o c_2 \o c_3,
  \end{align*}
  \begin{align*}
    (\mcu \o id) \circ \mcm = id = (id \o \mcu) \circ \mcm & \qquad
    c = \sum \mcm(c_1) c_2 = \sum c_1 \mcm(c_2),
  \end{align*}
  \begin{align*}
    \mm \circ (id \o \mci) \circ \mcm = \eta \circ \mcu = \mm \circ (\mci \o id) \circ \mcm & \qquad
    \mcu (c) = \sum c_1 \mci(c_2) = \sum \mci(c_1) c_2,
  \end{align*}

  где $ \eta $ --- единица $ \K[G] $, $ \mm $ --- умножение в $ \K[G] \o \K[G] $.

  \begin{definition}
    Супералгебра $ С $ вместе с коумножением, коединицей и антиподом,
    удовлетворяющими аксиомам~\ref{coassoc},~\ref{counit},~\ref{antipode}
    называется супералгеброй Хопфа.
  \end{definition}

  Таким образом, имеем дуальность категорий аффинных групповых суперсхем
  и супералгебр Хопфа.

  \begin{definition}
    Пусть $ С $ --- супералгебра Хопфа. Суперидеал $ I $ называется суперидеалом Хопфа,
    если $ \mcm (I) \subseteq C \o I + I \o C, ~I \subseteq \M = \ker\mcu, ~\mci(I) \subseteq I $.
  \end{definition}

  Аналогично топологии, определенной на аффинной суперсхеме
  в пункте~\ref{closed subfunctors}, на аффинной групповой суперсхеме
  топология определяется замкнутыми подфункторами $ V(I) $, соответствующими
  супеидеалам Хопфа.
  
\end{subsection}

\begin{subsection}{Суперкоалгебры и суперкомодули}
  Для простоты изложения материала супералгебры Хопфа были определены как
  объекты, дуальные аффинным групповым суперсхемам. Можно было сначала
  определить супералгебры Хопфа как, изначально приняв аксиомы
  ~\ref{coassoc},~\ref{counit},~\ref{antipode}. В этом разделе все-таки будут
  приведены некоторые стандартные понятия. Подробное изложение для случая
  обычных, неградуированных систем, можно найти в~\cite{sweedler}.

  Суперпространство, наделенное коумножением и коединицей с соответствующими
  аксиомами называется \defn{суперкоалгеброй} (соответственно, \defn{коалгеброй},
  если рассматривать неградуированные алгебры).
  \begin{definition}
    Суперкоалгебра $ С $ называется кокоммутативной, если
    $$ \mcm_C (c) = \sum c_1 \o c_2 = \sgn{c_1}{c_2} c_2 \o c_1. $$
  \end{definition}

  Супералгебра, которая в то же время
  является и суперкоалгеброй, называется \defn{супербиалгеброй}. Таким образом,
  супералгебра Хопфа --- супербиалгебра с антиподом.
  %
  \begin{definition}
    $ V $ называется правым суперкомодулем над суперкоалгеброй $ С $, если задано
    линейное отображение $ \mca: V \to V \o C $, называемое кодействием
    суперкоалгебры $ С $ на $ V $, которое сохраняет градуировку и
    для которого коммутативны следующие диаграммы коммутативны:

    \begin{equation}
      \begin{diagram}
        \node{C}
          \arrow[2]{e,t}{\mca}
          \arrow{s,l}{\mca}
        \node[2]{V \o C}
          \arrow{s,r}{\mca \o id} \\
        \node{V \o C}
          \arrow[2]{e,t}{id \o \mcm}
        \node[2]{V \o C \o C}
      \end{diagram}
      %
      \begin{diagram}
        \node{V \o C}
          \arrow[2]{e,t}{id \o \mcu}
        \node[2]{V \o \K} \\
        \node{V}
          \arrow{n,l}{\mca}
          \arrow{nee,b,<>}{\simeq}
      \end{diagram}      
    \end{equation}
  \end{definition}
  %
  Любая суперкоалгебра может быть наделена структурой суперкомодуля над собой,
  тогда кодействием является коумножение.
  %
  Пусть $ V, W $ --- (правые) суперкомодули над
  суперкоалгебрами $ C $ и $ B $ соответственно. Тензорное произведение
  $ C \o B $ наделяется структурой суперкоалгебры по правилу
  $ \mcm_{C \o B} (c) = (c \o b) = \sum \sgn{b_1}{c_2} (c_1 \o b_1) \o (c_2 \o b_2) $,
  где $ \mcm_C (c) = \sum c_1 \o c_2, ~\mcm_B (b) = \sum b_1 \o b_2 $.
  Более того, суперпространство $ V \o W $ будет $ C \o B $-суперкомодулем
  относительно кодействия $ \mca_{v \o W } (v \o w) =
  \sum \sgn{w_1}{c_2} (v_1 \o w_1) \o (c_2 \o b_2) $, где
  $ \mca_V (v) = \sum v_1 \o c_2, ~\mca_W (w) = \sum w_1 \o b_2 $.

  Если $ \p: C \to C' $ --- гомоморфизм суперкоалгебр, то произвольный
  $ C $-суперкомодуль будет и $ C' $-суперкомодулем относительно
  кодействия $ (id \o \p)_{\mca_V} $. Если $ C $ --- супербиалгебра, то
  отображение $ \mm: C \o C \to C $, индуцированное умножением,
  является гомоморфизмом суперкоалгебр. В частности, если
  $ V, W $ --- левые $С$-суперкомодули, то мы получаем диагональное
  кодействие $ (id \o \mm)_{\mca_{V \o W}} $ супербиалгебры $ C $ на $ V \o W $.
  Более того $ V \o W $ и $ W \o V $ изоморфны как $C$-суперкомодули относительно
  изоморфизма перестановки $ t: v \o w \mapsto \sgn{v}{w} w \o v, ~v \in V, ~w \in W $.

\end{subsection}

  %
  \newpage
  \section{Супералгебры распределений и их связь с супералгебрами Ли}
    \input{./source/03-superalbras-of-distributions.tex}

  %
  \newpage
  \section{Связные аффиные групповые суперсхемы}
    \begin{subsection}{}
  \begin{definition}
    Алгебраическая аффинная групповая суперсхема $ G = \SSp A $ называется
    псевдосвязной, если $ \bigcap_{n \geqslant 0} \M^n = 0 $, где $ \M = \ker\mci_A $.
  \end{definition}

  \begin{lemma}
    Пусть $ G $ --- алгебраическая аффинная групповая суперсхема. Суперидеал
    $ I = \bigcap_{n \geqslant 0} \M^n $ является суперидеалом Хопфа, а
    аффинная групповая суперподсхема $ G^{[0]} = V(I) $ нормальна и связна.
    \proof {
      По определению $ \mci_A(\M) = \M $.
      $$
        \mcm_A(\M^n) \subseteq \sum_{0 \leqslant i \leqslant} \M^i \o \M^{n-i}
        \subseteq \bigcap_{0 \leqslant i \leqslant} (\M^i \o A + A \o \M^{n-i})
      $$
      $$
        \mcm_A(I) \subseteq \bigcap_{n \geqslant 0} \mcm_A(\M^n) \subseteq
        \bigcap_{n \geqslant 0} (\M^n \o A + A \o \M^n) = I \o A + A \o I.
      $$
      \qedhere
    }
  \end{lemma}

  $ G^{[0]} $ называется \defn{псевдосвязной компонентой} $ G $. Очевидно, что
  $ G $ псевдосвязна тогда и только тогда, когда $ G = G^{[0]} $.

  \begin{lemma}[Теорема Крулля о пересечении]
    Пусть $ A $ --- конечнопорожденная коммутативная супералгебра и $ V $ ---
    конечнопорожденный $ A $-суперкомодуль. Для любого суперидела $ I \subseteq A $
    $ \bigcap_{n \geqslant 0} I^n V = {v \in V ~| ~\exists ~x \in I_0 : (1 - x)v = 0} $.
  \end{lemma}

  \begin{proposition}\label{exact sequence for Lie}
    Пусть $ \pi : G \to H $ --- эпиморфизм алгебраических аффинных групповых суперсхем.
    Если $ \charc\K = 0 $, то порожденная эпиморфизмом короткая последовательность
    супералгебр Ли
    $$ 0 \to \Lie(\ker\pi) \to \Lie(G) \xrightarrow{d\pi} \Lie(H) \to 0 $$
    является точной.
  \end{proposition}

  \begin{lemma}\label{subgroups iff Lie subalgebras}
    Пусть $ G $ --- алгебраическая аффинная груповая суперсхема, $ H_1, H_2 $ ---
    cуперподсхемы $ G $, $ H_1 $ псевдосвязна. Тогда
    $ H_1 \subseteq H_2 \iff \Dist(H_1) \subseteq \Dist(H_2) $, а если $ \charc\K = 0 $,
    то $ H_1 \subseteq H_2 \iff \Lie(H_1) \subseteq \Lie(H_2) $
  \end{lemma}

  \begin{lemma}
    Если $ G $ псевдосвязна или связна, то из $ \Lie(G) = 0 $ следует $ G = E $.
    В частности, если $ \charc\K = 0 $ и $ G $ алгебраическая, то $ G^{(0)} = G^{[0]} $.
  \end{lemma}

  Это важное утверждение позволяет пользоваться при рассмотрении
  алгебраических аффинных групповых суперсхем в случае поля нулевой характеристики
  использовать опредения связности и псевдосвязности как эквивалентные.

\end{subsection}


\begin{subsection}{}
  \begin{definition}
    Подфунктор $ \Z(G) $ групового $ \K $-функтора $ G $ называется центральным,
    если $ H $ -- подфунктор в $ G $ и $ \forall A \in \SAlg \; H(A) $ --
  \end{definition}


  \begin{proposition} \label{Z(G) closed in G}
    Пусть $ G $ --- аффинная групповая суперсхема.
    $ \Z(G) $ --- замкнутая аффинная групповая подсуперсхема в $ G $.
    \proof {

      \qedhere
    }
  \end{proposition}

  \begin{proposition} \label{Lie(Z(G)) = Z(Lie(G))}
    Если $ G $ связна, $ \charc K = 0 $, то $ \Lie(\Z(G)) = \Z(\Lie(G)) $.
    \proof {

      \qedhere
    }
  \end{proposition}

  \begin{theorem} \label{Exists H in G: Lie(H) = I}
    Пусть $ \charc K = 0 $, $ G $ --- связная аффинная групповая суперсхема, \\
    $ I $ --- максимальный абелев суперидеал в $ \Lie(G) $.
    Существует $ H \lhd G : \Lie(H) = I $.
    \proof {
      Обозначим $ L = \Lie(G) $.
      Доказательство проведем индукцией по $ \dim L $. Предположим,
      что если $ H $ --- связная аффинная групповая суперсхема и $ \dim \Lie(H) < \dim L $,
      то утверждение выполнено для $ H $.

      Рассмотрим действие $ \Ad: G \to \GL(I) $, $ \ker \Ad = R $.
      Пусть $ J = \Lie(R) = \{ x \in L | [x, I] = 0 \} $.
      Очевидно, $ I \subseteq J $.

      Если $ \dim J \leqslant \dim L $, то по предположению индукции утверждение выполнено для $ R $,
      т.е. $ \exists \; H \lhd R : \Lie(H) = I $. Поскольку $ H \lhd R $
      и $ R \lhd G $ как ядро $ \Ad $, то $ H \lhd G $, следовательно, утверждение выполнено для $ G $.

      Рассмотрим случай $ \dim J = \dim L $.
      % J = L
      Т.к. $ G $ алгебраическая, то $ \dim L < \infty \hence J = L $.
      % I ⊆ Z(L)
      Отсюда следует, что $ [L, I] = 0 $, а в силу определения центра $ I \subseteq \Z(L) $.
      % I = Z(L)
      По условию $ I $ --- максимальный суперидеал $ \hence $ $ I $ не может быть
      собственным подмножеством $ \hence $ $ I = \Z(L) $.
      % I = Lie(Z(G))
      По лемме \ref{Lie(Z(G)) = Z(Lie(G))} получаем, что $ I = \Lie(\Z(G)) $, а $ \Z(G) \lhd G $.
    \qedhere
    }
  \end{theorem}

\end{subsection}


  %
  \newpage
  \section{Разрешимость аффиных групповых суперсхем}
    Перед тем, как дать определение разрешимой аффинной групповой суперсхемы,
сначала необходимо определить коммутаторную суперподсхему. Нижеизложенное
повторяет описание коммутанта для обычных схем (\cite{waterhouse}, глава 10),
но более подробно.

Пусть $ G $ - аффинная групповая суперсхема над полем $ \K $.
Определим морфизм функторов $ \pi: G \x G \to G $, переводящее
$ (g, h) $ в коммутатор $ g h g^{-1} h^{-1} $. Ему соответствует коморфизм
$ \pi^*: \K[G] \to \K[G] \o \K[G] $. Обозначим $ I_1 = \ker\pi^* $.
Как ядро гомоморфизма супералгебр Хопфа $ I_1 $ --- суперидеал Хопфа.
Аналогично для $ n \in \N $ имеем морфизм
$ \pi_n: (g_1, h_1, \ldots, g_n, h_n) \mapsto g_1 h_1 g_1^{-1} h_1^{-1} \cdots g_n h_n g_n^{-1} h_n^{-1} $
и коморфизм $ \pi_n^*: \K[G] \to \K[G]^{\o 2n} $. $ I_n = \ker\pi_n^* $.

\begin{proposition}
  $ I_{n+1} \subseteq I_{n} $.
\end{proposition}

Обозначим $ I = \bigcap_{n \geqslant 0} I_n $.

\begin{proposition}
  $ I = \bigcap I_n $ -- суперидеал Хопфа.
  \proof {
    Для любого $ n \in \N \quad I_n $ --- суперидеал Хопфа, то есть
    $$ I \subseteq \ker\mcu, \quad
      \mcm(I_n) \subseteq \K[G] \o I_n + I_n \o \K[G], \quad
      \mci(I_n) \subseteq I.
    $$
    Очевидно, что $ I \subseteq \ker\mcu $ и $ \mci(I) \subseteq I $.

    Рассмотрим $ I_{2n} $. ~$ \pi_{2n} = \mm \circ (\pi_n \x \pi_n) $,
    поэтому можно записать дуальную коммутативную диаграмму:
    \begin{align*}
      \begin{diagram}
        \node{G^{2n} \x G^{2n}}
          \arrow{e,t}{\pi_n \x \pi_n}
          \arrow{se,b}{\pi_{2n}}
        \node{G \x G}
          \arrow{s,r}{\mm} \\
        \node[2]{G}
      \end{diagram} \qquad
      \begin{diagram}
        \node{\K[G]^{\o 2n} \o \K[G]^{\o 2n}}
        \node{\K[G] \x \K[G]}
          \arrow{w,t}{\pi_n^* \o \pi_n^*} \\
        \node[2]{\K[G]}
          \arrow{nw,b}{\pi_{2n}^*}
          \arrow{n,r}{\mcm}
      \end{diagram}
    \end{align*}
    Отсюда получаем, что $ \pi_{2n}^* = (\pi_n^* \x \pi_n^*) \circ \mcm $.
    Если $ \mcm(\p) = 0 $, то $ (\pi_n^* \x \pi_n^*)(\mcm(\p)) = 0 $, поэтому
    \begin{equation}
      \mcm(I_{2n}) \subseteq \ker(\pi_n^* \x \pi_n^*) = I_n \o \K[G] + \K[G] \o I_n.
    \end{equation}
    $ I = \bigcap I_n = \bigcap I_{2n} ~\hence ~\mcm(I) = \mcm(\bigcap I_{2n})
    \subseteq \bigcap (I_n \o \K[G] + \K[G] \o I_n) =
    \bigcap I_n \o \K[G] + \K[G] \o \bigcap I_n  = I \o \K[G] + \K[G] \o I $,
    что и требовалось для суперидеала Хопфа.
    \qedhere
  }
\end{proposition}

\begin{proposition}
  Замкнутая суперподсхема $ V(I) $, определяемая суперидеалом Хопфа $ I $,
  является наименьшей замкнутой суперподсхемой, содержащая произведения
  любых коммутаторов.
\end{proposition}

\begin{definition}
  Замкнутая подгруппа $ V(I) = G' $, определяемая суперидеалом Хопфа $ I $,
  называется коммутантом аффинной групповой суперсхемы $ G $.
\end{definition}

\begin{proposition}
  $ G' $ --- нормальная суперподсхема аффинной групповой суперсхемы $ G $.
\end{proposition}

Стандартным образом определим $n$-й коммутант $ G^{(n)} $.

\begin{definition}
  Будем называть аффинную групповую суперсхему $ G $ разрешимой,
  если $ G^{(n)} $ тривиальна для некоторого $ n $.
\end{definition}

\begin{theorem}
  Пусть $ G $ алгебраическая. Если $ G $ связна, то и $ G' $ связна.
\end{theorem}


  %
  \newpage
  \section{Аналог теоремы Каца}
    \begin{lemma}\label{Lie(Gev) = L_0}
  Обозначим $ \Lie(G) = L = L_0 \oplus L_1 $. $ \Lie(G_{ev}) = L_0 $.
%  \proof {
%    \qedhere
%  }
\end{lemma}

\begin{lemma} \label{abelian G and Lie(G)}
  Аффинная групповая суперсхема $ G $ абелева $ \iff $ $ \Lie(G) $ абелева.
%  \proof {
%    Достаточно доказать, что $ \Dist(G) $ абелева $ \iff \K[G]^{*} $ кокоммутативна.
%    \qedhere
%  }
\end{lemma}

\begin{theorem}[Кац] \label{kac}
  Супералгебра Ли $ L = L_0 \oplus L_1 $ разрешима тогда и только тогда,
  когда разрешима алгебра Ли $ L_0 $.
\end{theorem}
Доказательство можно найти в статье \cite{kac}. Теперь все готово
для доказательства основной теоремы этой работы.

\begin{theorem}
  Пусть $ \charc K = 0 $, $ G $ - связная алгебраическая аффинная групповая суперсхема.
  $ G $ разрешима $ \iff $ $ \Lie(G)$ разрешима $ \iff $ $ G_{ev} $ разрешима.

  \proof{
    \begin{trivlist}
      \item а)
        Предположим, что $ G $ разрешима, т.е. для некоторого $ n \in \N $
        \begin{equation}\label{g condition}
          G ~\rhd ~G' ~\rhd ~G'' ~\rhd ~\ldots ~\rhd ~G^{(n)} = 1.
        \end{equation}
        % Exact sequence for first derived subgroup
        Рассмотрим абелев фактор $ G/G' $. Отображение $ \pi: G \to G/G' $ является
        эпиморфизмом алгебраических аффинных групповых суперсхем,
        следовательно по утверждению~\ref{exact sequence for Lie} имеем точную последовательность
        \begin{align*}
          0 \to \Lie(G') \to \Lie(G) \to \Lie(G/G') \to 0,
        \end{align*}
        Откуда получаем, что $ \Lie(G') \subseteq \Lie(G) $. Поскольку фактор
        $ G/G' $ абелев, то по лемме~\ref{abelian G and Lie(G)} $ \Lie(G/G') $ абелева,
        а значит и фактор $ \Lie(G) / \Lie(G') = \Lie(G/G') $ абелев.
        Рассматривая таким образом все факторы цепочки~(\ref{g condition}), получаем цепочку
        \begin{align*}
          \Lie(G) ~\rhd ~\Lie(G') ~\rhd ~\Lie(G'') ~\rhd ~\ldots ~\rhd ~\Lie(G^{(n-1)} ~\rhd 0,
        \end{align*}
        в которой все факторы абелевы, то есть $ \Lie(G) $ разрешима.
        \newline

        Обратно, предположим, что $ \Lie(G) $ разрешима, то есть существует цепочка
        \begin{equation} \label{lie condition}
          \Lie(G) ~\rhd ~I_1 ~\rhd ~I_2 ~\rhd ~\ldots ~\rhd ~I_n = 0.
        \end{equation}
        Рассмотрим $ I_{n-1} $. Т.к. $ I_n = 0 $,  то $ I_{n-1} $ ---
        максимальный суперидеал $ \Lie(G) $. При этом он абелев, как и все суперидеалы этой цепочки.
        Тогда по теореме~\ref{Exists H in G: Lie(H) = I} существует нормальная суперподсхема
        $ H_{n-1} \lhd G: \Lie(H_{n-1}) = I_{n-1} $. Суперидеал $ I_{n-1} $ абелев $ ~\hence $
        ~$ \Lie(H_{n-1}) $ абелева $ ~\hence $ абелевы ~$ H_{n-1} $ и $ G/H_{n-1} $.

        Теперь рассмотрим $ \Lie(G) / I_{n-1} = \Lie(G/H_{n-1}) $. $ I_{n-1} $ ---
        максимальный абелев суперидеал $ \Lie(G/H_{n-2}) ~\hence
        ~\exists ~H_{n-2} ~\lhd ~G/H_{n-1}: ~\Lie(H_{n-2}) = I_{n-2} $.
        Таким образом, мы получили $ \Lie(H_{n-1}) \lhd \Lie(H_{n-2}) $,
        а по лемме~\ref{subgroups iff Lie subalgebras} получаем, что $ H_{n-1} \lhd H_{n-2} $.
        %
        Аналогичным образом продолжая разбирать цепочку~(\ref{lie condition}) вверх, получаем цепочку
        \begin{align*}
          G ~\rhd ~H_1 ~\rhd ~\ldots ~\rhd ~H_{n-2} ~\rhd ~H_{n-1} ~\rhd ~E,
        \end{align*}
        в которой все факторы абелевы, то есть $ G $ разрешима.

        Таким образом, мы доказали, что $ G $ разрешима $ \iff $ $ \Lie(G)$ разрешима.

      \item б) По теореме~\ref{kac} $ \Lie(G) $ разрешима тогда и только тогда,
        когда разрешима $ \Lie(G)_0 $. По лемме~\ref{Lie(Gev) = L_0} $ \Lie(G)_0 = \Lie(G_{ev}) $,
        а из первой части доказательства следует, что $ \Lie(G_{ev}) $ разрешима
        тогда и только тогда, когда $ G_{ev} $ разрешима.
        Таким образом доказана вторая эквивалентность, а с ней и вся теорема.
        \qedhere
    \end{trivlist}
  }
\end{theorem}

 
  %
  \newpage
  \section*{Заключение}
    \include{./source/conclusion.tex}

  \bibliography{biblio}
\end{document}